%-------------------------------------------------------------------------------
%	PACKAGES AND OTHER DOCUMENT CONFIGURATIONS
%-------------------------------------------------------------------------------

\documentclass[11pt]{article}

% Packages
% packages to load
\usepackage{amssymb} %
\usepackage[utf8]{inputenc}
%\usepackage[margin=1in]{geometry}
\usepackage{graphics,graphicx} % jpg and pdf figures
%\usepackage{sidecap}
\usepackage{amsmath}
\usepackage{verbatim} % block comments
\usepackage{multicol, caption} % multicolumn formating
\usepackage{lipsum} % needed for figures in multicol format
\usepackage{float} % for figure placement
\usepackage{natbib} % for bibtex referencing
\usepackage{cite} % cite with bibtex
\usepackage{aas_macros} % for journal referencing





% formatting
%%%%%%%%%%%%%%%%%%%%%%%%%%%%
%%%%%% Page dimensions %%%%%
%%%%%%  DO NOT CHANGE  %%%%%
%%%%%%%%%%%%%%%%%%%%%%%%%%%%

\textheight=247mm
\textwidth=180mm
\topmargin=-7mm
\oddsidemargin=-10mm
\evensidemargin=-10mm
\parindent 10pt

% Citation format
\bibliographystyle{apj}
\citestyle{aa}



\bibliographystyle{apj}
\usepackage{float} % for figure placement
\usepackage{natbib} % for bibtex referencing
\usepackage{cite} % cite with bibtex
\usepackage{aas_macros} % for journal referencing
\usepackage{hyperref} % for including urls

%-------------------------------------------------------------------------------
%	NAME AND CLASS SECTION
%-------------------------------------------------------------------------------

\newcommand{\hmwkTitle}{HW 11} % Assignment title
\newcommand{\hmwkDueDate}{Tuesday, Oct. 28} % Due date
\newcommand{\hmwkClass}{Stat 860} % Course/class
\newcommand{\hmwkClassTime}{4:00 PM} % Class/lecture time
\newcommand{\hmwkClassInstructor}{Grace Wahba} % Teacher/lecturer
\newcommand{\hmwkAuthorName}{Elijah Bernstein-Cooper} % Your name

%-------------------------------------------------------------------------------
%	TITLE PAGE
%-------------------------------------------------------------------------------

\title{\vspace{0in}
    \textmd{\textbf{\hmwkClass:\ \hmwkTitle}}\\
    \normalsize\vspace{0.1in}\small{Due\ on\ \hmwkDueDate}\\
    \vspace{0.1in}\large{\textit{\hmwkClassInstructor\ \hmwkClassTime}}
    \vspace{0.2in}}

\author{\textbf{Elijah Bernstein-Cooper}}
\date{\today} % Insert date here if you want it to appear below your name

%-------------------------------------------------------------------------------

\begin{document}

\maketitle
%\newpage

%===============================================================================
%-------------------------------------------------------------------------------
%	PROBLEM 1
%-------------------------------------------------------------------------------
\begin{homeworkProblem}

    Abstract for \citet{lindner:vera-ciro:murray:stanimirovic:2014}:

    \begin{quote}

        We present a new algorithm, named Autonomous Gaussian Decomposition
        (AGD), for automatically decomposing spectra into Gaussian components.
        AGD uses derivative spectroscopy and machine learning to provide
        optimized guesses for the number of Gaussian components in the data,
        and also their locations, widths, and amplitudes. We test AGD and find
        that it produces results comparable to human-derived solutions on 21cm
        absorption spectra from the 21cm SPectral line Observations of Neutral
        Gas with the EVLA (21-SPONGE) survey. We use AGD with Monte Carlo
        methods to derive the HI line completeness as a function of peak
        optical depth and velocity width for the 21-SPONGE data, and also show
        that the results of AGD are stable against varying observational noise
        intensity. The autonomy and computational efficiency of the method over
        traditional manual Gaussian fits allow for truly unbiased comparisons
        between observations and simulations, and for the ability to scale up
        and interpret the very large data volumes from the upcoming Square
        Kilometer Array and pathfinder telescopes.
        
    \end{quote}

    URL: \url{http://arxiv.org/abs/1409.2840} \\

    \clearpage
    Abstract for \citet{yeow:azali:ow:wong:2005}:
    
    \begin{quote}
        
        The problem of differentiating spectral data to yield the third and
        fourth derivatives is converted into one of solving an integral
        equation of the first kind. This equation is solved by Tikhonov
        regularization. The method of General Cross Validation is used to guide
        the choice of the regularization parameter that keeps noise
        amplification under control. The performance of this route to third and
        fourth derivative spectra is demonstrated by applying it to a number of
        published spectra. A computational problem associated with General
        Cross Validation has been identified.

    \end{quote}

    URL: \url{http://www.hindawi.com/journals/isrn/2011/164564/abs/} \\

    Abstract for \citet{chartrand:2011}:

    \begin{quote}

        We consider the problem of differentiating a function specified by
        noisy data. Regularizing the differentiation process avoids the noise
        amplification of finite-difference methods. We use total-variation
        regularization, which allows for discontinuous solutions. The resulting
        simple algorithm accurately differentiates noisy functions, including
        those which have a discontinuous derivative.

    \end{quote}

    URL:
    \url{http://www.sciencedirect.com/science/article/pii/S0039914005003267} \\

\end{homeworkProblem}
%===============================================================================

    \bibliography{refs}

\end{document}

