%-------------------------------------------------------------------------------
%	PACKAGES AND OTHER DOCUMENT CONFIGURATIONS
%-------------------------------------------------------------------------------

\documentclass[11pt]{article}

% Packages
% packages to load
\usepackage{amssymb} %
\usepackage[utf8]{inputenc}
%\usepackage[margin=1in]{geometry}
\usepackage{graphics,graphicx} % jpg and pdf figures
%\usepackage{sidecap}
\usepackage{amsmath}
\usepackage{verbatim} % block comments
\usepackage{multicol, caption} % multicolumn formating
\usepackage{lipsum} % needed for figures in multicol format
\usepackage{float} % for figure placement
\usepackage{natbib} % for bibtex referencing
\usepackage{cite} % cite with bibtex
\usepackage{aas_macros} % for journal referencing





% formatting
%%%%%%%%%%%%%%%%%%%%%%%%%%%%
%%%%%% Page dimensions %%%%%
%%%%%%  DO NOT CHANGE  %%%%%
%%%%%%%%%%%%%%%%%%%%%%%%%%%%

\textheight=247mm
\textwidth=180mm
\topmargin=-7mm
\oddsidemargin=-10mm
\evensidemargin=-10mm
\parindent 10pt

% Citation format
\bibliographystyle{apj}
\citestyle{aa}



%-------------------------------------------------------------------------------
%	NAME AND CLASS SECTION
%-------------------------------------------------------------------------------

\newcommand{\hmwkTitle}{HW 7} % Assignment title
\newcommand{\hmwkDueDate}{Thursday, Oct. 9} % Due date
\newcommand{\hmwkClass}{Stat 860} % Course/class
\newcommand{\hmwkClassTime}{4:00 PM} % Class/lecture time
\newcommand{\hmwkClassInstructor}{Grace Wahba} % Teacher/lecturer
\newcommand{\hmwkAuthorName}{Elijah Bernstein-Cooper} % Your name

%-------------------------------------------------------------------------------
%	TITLE PAGE
%-------------------------------------------------------------------------------

\title{\vspace{0in}
    \textmd{\textbf{\hmwkClass:\ \hmwkTitle}}\\
    \normalsize\vspace{0.1in}\small{Due\ on\ \hmwkDueDate}\\
    \vspace{0.1in}\large{\textit{\hmwkClassInstructor\ \hmwkClassTime}}
    \vspace{0.2in}}

\author{\textbf{Elijah Bernstein-Cooper}}
\date{\today} % Insert date here if you want it to appear below your name

%-------------------------------------------------------------------------------

\begin{document}

\maketitle
%\newpage

%===============================================================================
%-------------------------------------------------------------------------------
%	PROBLEM 1
%-------------------------------------------------------------------------------
\begin{homeworkProblem}

    We fit a thin plate spline model of noisy data with varying degrees of
    noise. We fit the spline model with generalized cross-validation
    minimization, and with too large of a smoothing parameter and too small of
    a smoothing parameter. With decreasing noise, the derived the minimizer of
    the Hilbert space, $\mathcal{H}$, will be determined with greater
    confidence. $\Lambda$ will increase with larger variation in the data, the
    residual sum of squares will increase, the degrees of freedom of the noise
    will increase, and the estimate of $\sigma$ will increase. Figures 1
    through 5 show resulting fits to data with different degrees of noise. For
    smaller noise in the data, the oversmoothed and undersmoothed fits to the
    data are more accurate than for larger noise in the data. This is due to
    the smaller smoothing required to reproduce the true function in less noisy
    data, and the code only scales the GCV estimate of the smoothing factor,
    thus the oversmoothed and undersmoothed fits will not be as extreme for
    noisier data.
    
        \begin{figure}[!ht]
            
            \begin{centering}
                \includegraphics[scale=0.8]{result_0001.png}

            \caption{Results from $\sigma = 0.001m$.}

            \end{centering}
        \end{figure}

        \begin{figure}[!ht]
            
            \begin{centering}
                \includegraphics[scale=0.8]{result_006325.png}

            \caption{Results from $\sigma = 0.006235m$.}

            \end{centering}
        \end{figure}
        \begin{figure}[!ht]
            
            \begin{centering}
                \includegraphics[scale=0.8]{result_01255.png}

            \caption{Results from $\sigma = 0.01255m$.}

            \end{centering}
        \end{figure}

        \begin{figure}[!ht]
            
            \begin{centering}
                \includegraphics[scale=0.8]{result_018775.png}

            \caption{Results from $\sigma = 0.018775m$.}

            \end{centering}
        \end{figure}

        \begin{figure}[!ht]
            
            \begin{centering}
                \includegraphics[scale=0.8]{result_025.png}

            \caption{Results from $\sigma = 0.025m$.}

            \end{centering}
        \end{figure}



\end{homeworkProblem}
%===============================================================================


\end{document}

