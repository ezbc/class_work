

% NOAOPROP.TEX -- template for NOAO STANDARD telescope proposals.
% Revision Feb 1 2001 -- for the Aug 2001 - Jan 2002 observing semester.
% For later semesters, the current form may be obtained from our Web
% page at http://www.noao.edu/noaoprop/noaoprop.html

%       STANDARD PROPOSAL FORM - DO NOT USE FOR SURVEY PROPOSALS

% DEADLINE FOR SUBMISSIONS: 11:59pm MST SATURDAY MARCH 31.

% This LaTeX template has been returned to you upon request through
% our Web-based proposal form.  The template has been customized for
% you based on the run information that you entered via the Web form.
% As an option you may complete this form locally and submit it by
% email following the instructions below.  If the run information is
% incomplete you should go back to the Web form, complete all the
% suggested run information and then obtain another customized template
% before proceeding.  In particular, Gemini run information including
% targets and guide stars must be completed through the Web form.
%
% We encourage Web submissions but if you prefer to submit this
% template by email, follow the instructions below.
%
%   1. Where/how/when to submit this form electronically:
%      Send THIS file (NOT the PostScript version) to
%      noaoprop-submit@noao.edu.  The proposal and any figures
%      must be submitted as separate email messages.  Figures can be
%      submitted only AFTER the proposal ID has been sent to you via
%      email. See the end of this file for further submission details.
%
%   2. Before submitting this form electronically, run it through latex
%      and print it out to make certain that it looks the way that you
%      wish the review panel to see it.
%
%   3. If the proposal is a thesis or if the principal investigator is
%      a graduate student, the student's faculty advisor must send a
%      letter citing the graduate student's observing experience and,
%      if the program is a thesis, how this particular observing
%      proposal fits into the overall thesis plans.  This letter must
%      be sent by the faculty advisor to noaoprop-letter@noao.edu
%      BEFORE the proposal deadline.  Graduate students proposing
%      thesis observations should consult NOAO policies concerning
%      thesis programs and travel support.
%
%   4. If you are planning to bring a Visitor Instrument, you must
%      send a separate letter or email (noaoprop-letter@noao.edu) prior
%      to Oct 15 to make sure we fully understand what will be involved
%      in interfacing your equipment to our telescopes.
%
%   5. QUESTIONS?  If you have questions about submitting your proposal
%      send email to noaoprop-help@noao.edu.  Information about
%      instrumentation and observing facilities at Gemini, CTIO, KPNO,
%      and the HET can be found at the NOAO Web site at
%      http://www.noao.edu/noaoprop/noaoprop.html.  Specific questions
%      about instrumentation or facilities at CTIO, KPNO, or the HET
%      can also be sent to the respective sites at ctio@noao.edu,
%      kpno@noao.edu, or het@noao.edu.   Gemini questions should be
%      sent to their Web page at http://helpdesk.gemini.edu/hdsupport/
%      or email may be sent to the US mirror scientists at
%      usgemini@noao.edu.
%
%   6. When your proposal is received at NOAO you will be sent an
%      automatic email message verifying its receipt along with a
%      proposal ID number.  If you do not receive this message within
%      15 minutes of the time you sent your proposal send email to
%      noaoprop-help@noao.edu for assistance.  You may track your
%      proposal processing by the proposal ID number at the following
%      web page:   http://www.noao.edu/cgi-bin/noaoprop/propstatus
% ___________________________________________________________________
% THE FORM STARTS HERE
%

% Please do not modify or delete this line.
\documentstyle[nprop24,11pt]{article}


% Please do not modify or delete this line.
\begin{document}

% Please do not modify or delete this line.
\proposaltype{Longterm}

%%%%%%%%%%%%%%%%%%%%%%%%%%%%%%%%%%%%%%%%%%%%%%%%%%%%%%%%%%%%%%%%%%%%%

% SCIENTIFIC CATEGORIES
%
% Please select a "scientific category" that best describes your
% program by uncommenting only ONE of the selections below.  Your
% \sciencecategory selection will be used to assign a review panel to
% your proposal.  DO NOT MODIFY THE SELECTION YOU UNCOMMENT.  A
% description of each of these categories is available on our Web page
% at http://www.noao.edu/noaoprop/help/scicat.html

% EXTRA-GALACTIC LIST (do not uncomment this line)
%\sciencecategory{Active Galaxies}
%\sciencecategory{Cosmology}
%\sciencecategory{Large Scale Struc.}
%\sciencecategory{Clusters of Galaxies}
%\sciencecategory{High Z Galaxies}
%\sciencecategory{Low Z Galaxies}
%\sciencecategory{Resolved Galaxies}
%\sciencecategory{Stellar Pops (EGAL)}
%\sciencecategory{EGAL - Other}

% GALACTIC/LOCAL GROUP LIST (do not uncomment this line)
\sciencecategory{Star Clusters}
%\sciencecategory{Stellar Pops (GAL)}
%\sciencecategory{HII Reg., PN, etc.}
%\sciencecategory{ISM}
%\sciencecategory{Star Forming Regions}
%\sciencecategory{Young Stellar Obj.}
%\sciencecategory{Massive Stars}
%\sciencecategory{Low Mass Stars}
%\sciencecategory{Stellar Remnants}
%\sciencecategory{Galactic - Other}

% SOLAR SYSTEM LIST (do not uncomment this line)
%\sciencecategory{Kuiper Belt Objects}
%\sciencecategory{Small Bodies \& Moons}
%\sciencecategory{Planets}
%\sciencecategory{Solar System - Other}

%%%%%%%%%%%%%%%%%%%%%%%%%%%%%%%%%%%%%%%%%%%%%%%%%%%%%%%%%%%%%%%%%%%%%%

% TITLE
%
% Give a descriptive title for the proposal in the \title command.
%
% Note that a title can be quite long; LaTeX will break the title into
% separate lines automatically.  If you wish to indicate line breaks
% yourself, do so with a `\\' command at the appropriate point in
% the title text.  Use both upper and lower case letters (NOT ALL CAPS).

\title{A Sample NOAO Telescope Proposal}

%%%%%%%%%%%%%%%%%%%%%%%%%%%%%%%%%%%%%%%%%%%%%%%%%%%%%%%%%%%%%%%%%%%%%%%

% INVESTIGATOR'S (PI AND CoI) INFORMATION BLOCKS
%
% Please give the PI's name (first name first followed by middle
% initial and last name), affiliation, department and complete mailing
% address, as well as an email address.  Also give a complete phone
% number, and a number for a fax machine if you have access to one.
% You must also indicate the principal investigator's status with one
% of the one-letter codes inside the \invstatus{} curly braces, as
% indicated below.
%
% The affil{}, \department{}, \address{} (use a comma separate list as
% needed), \city{}, \state{}, \zipcode{}, and \country{} (for non-US
% addresses) fields will be used together as your full postal mailing
% address.  Please be sure this information is complete.  Note that
% some institutions will not deliver postal mail if a department is
% not included in the postal mailing address.  Non-US addresses should
% include the country and any local postal codes.
%
% The fax number does not print on the form.
%
% For each CoI please include a name, affiliation, email address and
% investigator's status within the \begin{CoI} and \end{CoI} lines.
%
% For each \invstatus{} field, please fill in the appropriate
% investigator status code from the following list.  If the investigator
% is a graduate student, indicate "T" if this proposal is related to a
% thesis project, or "G" otherwise.  This code should represent the
% status of the individuals at the time of the proposal submission.
% This information is necessary to assist us with our required reporting
% to the NSF.
%
% \invstatus{P} % investigator has obtained PhD or doctorate
% \invstatus{T} % investigator is grad student, observing for thesis
% \invstatus{G} % investigator is grad student, not observing for thesis
% \invstatus{U} % investigator is an undergraduate student
% \invstatus{O} % investigator has other status (none of the above)
%
%
% DO NOT remove the \begin{PI} and \end{PI}.  Only one individual's name
% per \name field is allowed.

\begin{PI}
\name{George W. Bush}
\affil{Office of the President}
\department{}
\address{1600 Pennsylvania Avenue}
\city{Washington}
\state{DC}
\zipcode{20500}
\country{}
\email{president@whitehouse.gov}
\phone{(202) 456-1414}
\fax{(202) 456-2883}
\invstatus{O}
\end{PI}

\begin{CoI}
\name{Colin L. Powell}
\affil{Dept. of State}
\email{secretary@state.gov}
\invstatus{O}
\end{CoI}

\begin{CoI}
\name{J. Dennis Hastert}
\affil{U.S. House of Representatives}
\email{}
\invstatus{O}
\end{CoI}

%\begin{CoI}
%\name{}
%\affil{}
%\email{}
%\invstatus{}
%\end{CoI}

%\begin{CoI}
%\name{}
%\affil{}
%\email{}
%\invstatus{}
%\end{CoI}



% You can supply more CoI blocks, but only the first six will be
% printed in the main body of the proposal - others will be printed
% in the right margin.  All investigator names will be available to
% the review panel and stored in our database.

%%%%%%%%%%%%%%%%%%%%%%%%%%%%%%%%%%%%%%%%%%%%%%%%%%%%%%%%%%%%%%%%%%%%%%

% ABSTRACT
%
% Give a general abstract of the scientific justification appropriate
% for a non-specialist.  Write the abstract between the \begin{abstract}
% and \end{abstract} lines.  Limit yourself to approximately 175 words.
% Abstracts of accepted proposals will be made publicly available.

% DO NOT remove the \begin{abstract} and \end{abstract} lines.

\begin{abstract}
This sample proposal offers tips on how to prepare your telescope
proposal for observing on facilities available through NOAO. 
With the NOAO Proposal Form, you can apply for time on the 
Gemini North and South Telescopes, the  Hobby-Eberly
Telescope, the 6.5-m telescope of the MMT Observatory, and the telescopes
at the Cerro Tololo Inter-American Observatory and the Kitt Peak
National Observatory.  

Your abstract is the review panel's window into your proposal: the abstract
provides an initial impression about your proposal and it is also
what panel members refer to at the review meeting to remind themselves
about the content of your proposal.  Take advantage of the opportunity
to give the panel members an understandable and concise summary of what 
you want to do, and why.  Write your abstract so that non-specialists can
quickly understand why the observations you want to make are important.
\end{abstract}

%%%%%%%%%%%%%%%%%%%%%%%%%%%%%%%%%%%%%%%%%%%%%%%%%%%%%%%%%%%%%%%%%%%%%%

% SUMMARY OF OBSERVING RUNS REQUESTED
%
% List a summary of the details of the observing runs being requested,
% for UP TO SIX runs.  The parameters for each run are segregated
% between \begin{obsrun} and \end{obsrun} lines.  Please be sure
% that the information is isolated properly for each run.
%
%   \begin{obsrun}
%   \telescope{}        % For example, \telescope{KP-4m}
%   \instrument{}       % For example, \instrument{ECHUV + T2KB}
%   \numnights{}        % For example, \numnights{6}
%   \lunardays{}        % For example, \lunardays{grey}
%   \optimaldates{}     % For example, \optimaldates{Sep - Nov}
%   \acceptabledates{}  % For example, \acceptabledates{Aug - Jan}
%   \end{obsrun}
%
% The following telescope identifiers MUST be used in the \telescope{}
% field.  Some of the telescope identifiers must include an observatory
% code as well since the same form can be used to submit a single 
% proposal for all facilities available through NOAO.
%
% Gemini: GEM-N, GEM-NQ, GEM-SQ
% Hobby-Eberly Telescope: HET
% MMT:  MMT
% CTIO: CT-4m, CT-1.5m, YALO, CT-0.9m
% KPNO: KP-4m, WIYN, KP-2.1m, W-0.9m
%
% Select the instrument and detector identifiers from the list on our
% Web page at http://www.noao.edu/noaoprop/help/facilities.html.
% The correct codes MUST be used to ensure your correct
% instrument + detector combination.
%
% \numnights should give the number of nights of the run (for queue
% runs use fractional 10-hour equivalent nights (e.g. 3 hours = 0.3 nights),
% Formats such as 5x0.5 are acceptable.
%
% \lunardays should contain the word "darkest", "dark", "grey", or
% "bright", which in turn reflects the number of nights from new moon
% where darkest<=3, dark<=7, grey<=10, bright<=14.  Particular lunar 
% phase requirements dictated by the science program (e.g., "<=12", 
% "+9, -6", or "full moon more than 2 hours away from Taurus") should
% be noted in the "scheduling constraints or non-usable dates" section 
% below.
%
% \optimaldates should contain the range of OPTIMAL months, as shown 
% below.
%
% \acceptabledates should give the range of ACCEPTABLE months (i.e.,
% you would not accept time outside those limits).
% NOTE THAT DUE TO INSTRUMENT BLOCKING RESTRICTIONS YOU SHOULD MAKE 
% THIS RANGE AS GENEROUS AS POSSIBLE.
%
% For QUEUE-SCHEDULED observations, you may set the date range to the 
% full semester range and set \lunardays to the brighest moon your
% observations could tolerate if the program were scheduled classically.
%
% To enter the acceptable and optimal date ranges, please use two
% dash-separated months with 3-letter abbreviations for the month
% (Jan, Feb, Mar, Apr, May, Jun, Jul, Aug, Sep, Oct, Nov, Dec).
% For example:  \optimaldates{Nov - Dec}.
% We appreciate your help in not using vague range specifications
% like "October dark run" or "mid-January" which will require human
% intervention.
%
% FOR LONGTERM STATUS PROPOSALS SPECIFY ONLY THE RUNS FOR THE CURRENT
% SEMESTER, AND NOT FOR ANY SUBSEQUENT SEMESTERS.

% DO NOT remove any of the \begin{obsrun} and \end{obsrun} blocks, 
% even if the blocks are empty.

\begin{obsrun}
\telescope{HET}
\instrument{LRS}
\numnights{2}
\lunardays{grey}
\optimaldates{Aug - Jan}
\acceptabledates{Aug - Jan}
\end{obsrun}

\begin{obsrun}
\telescope{MMT}
\instrument{BCHAN}
\numnights{3x0.5}
\lunardays{dark}
\optimaldates{Dec - Jan}
\acceptabledates{Oct - Jan}
\end{obsrun}

\begin{obsrun}
\telescope{KP-4m}
\instrument{RCSPM + T2KB}
\numnights{3}
\lunardays{grey}
\optimaldates{Sep - Nov}
\acceptabledates{Aug - Jan}
\end{obsrun}

\begin{obsrun}
\telescope{WIYN}
\instrument{HYDRR + T2KC}
\numnights{2}
\lunardays{grey}
\optimaldates{Aug - Jan}
\acceptabledates{Aug - Jan}
\end{obsrun}

\begin{obsrun}
\telescope{CT-1.5m}
\instrument{CSPEC + L1K}
\numnights{5}
\lunardays{grey}
\optimaldates{Aug - Oct}
\acceptabledates{Aug - Sep}
\end{obsrun}

\begin{obsrun}
\telescope{}
\instrument{}
\numnights{}
\lunardays{}
\optimaldates{}
\acceptabledates{}
\end{obsrun}


% If there are scheduling constraints or non-usable dates for any of
% the runs specified, (i.e., other than the default lunar phase 
% requirements or when your object is up) please give the dates by 
% filling in the curly braces in \unusabledates{}.  Note here if you 
% are requesting runs in an "either/or" situation, e.g. run 1 or run 2, 
% but not both. This is also the place to advise us of any special 
% constraints which affect the scheduling of your observing run (e.g. 
% "schedule run #1 before run #2" or "run dates must be coordinated 
% with HST observations").

% Please limit your text to 4 lines on the printed copy.

\unusabledates{Please avoid Nov. 7 (Election Day) \\ The MMT run should follow the KP-4m run by at least a week.  \\  The CT-1.5m run must be scheduled Aug. 28 - Sept. 1 to catch the eclipse.}

%%%%%%%%%%%%%%%%%%%%%%%%%%%%%%%%%%%%%%%%%%%%%%%%%%%%%%%%%%%%%%%%%%%%%%

% In the following "essay question" sections, the delimiting pieces of
% markup (\justification, \expdesign, etc.) act as LaTeX \section*{}
% commands.  If the author wanted to have numbered subsections within
% any of these, LaTeX's \subsection could be used.
%
% DO NOT REDUCE THE FONT SIZE, and do not otherwise fiddle with the
% format to get more on a page.  We will reset any changes back to the
% default font.

% SCIENTIFIC JUSTIFICATION
%
% Give the scientific justification for the proposed observations.
% This section should consist of paragraphs of text followed by any
% references and up to three figures and captions.  Be sure to include
% overall significance to astronomy.  THE SCIENTIFIC JUSTIFICATION
% SHOULD BE LIMITED TO ONE PAGE (the review panels have requested that
% we not send them more than one page), with up to two additional pages
% for references, figures (no more than three), and captions. 

% If you wish to use our "reference" environment, follow the following
% example (journal commands are compatible with AASTeX v4.0):
%
%\begin{references}
%\reference Armandroff \& Massey 1991 \aj, 102, 927.
%\reference Berkhuijsen \& Humphreys 1989 \aap, 214, 68.
%\reference Massey 1993 in Massive Stars: Their Lives in the 
% Interstellar Medium (Review), ed. J. P. Cassinelli and E. B. 
% Churchwell, p. 168.
%\reference Massey \& Armandroff 1999, in prep.
%\end{references}

% Only EPS figures may be included with your proposal.  In order to
% include an EPS plot, you should use the LaTeX "figure" environment.
% The plot file is included with the \plotone{FILENAME} command; two
% side-by-side plot files can be included by typing
% \plottwo{FILENAME1}{FILENAME2}.  Use \caption{} to specify a caption.
% The \epsscale{} command can be used to scale \plotone plots if they
% appear too large on the printed page.  Contact us if you have any
% figure questions or encounter any problems with figures
% (noaoprop-help@noao.edu).
%
% \begin{figure}
% \epsscale{0.85}
% \plotone{sample.eps}
% \caption{Sample figure showing important results.}
% \end{figure}
%
% If you need to rotate or make other transformations to a figure, you
% may use the \plotfiddle command:
% \plotfiddle{PSFILE}{VSIZE}{ROTANG}{HSCALE}{VSCALE}{HTRANS}{VTRANS}
% \plotfiddle{sample.eps}{2.6in}{-90.}{32.}{32.}{-250}{225}
% where HSCALE and VSCALE are percentages and HTRANS and VTRANS are
% in PostScript units, 72 PS units = 1 inch.
%
% Note that the Web form provides several useful and simple figure 
% options.

\sciencejustification
The scientific justification should explain the overall goals of
your program in the context of your field, as well as the importance
of your program to astronomy.
Writing a good scientific justification is an art.  It takes
skill and practice.  And it requires a good scientific idea.
This last you must supply but a few general guidelines
about proposal writing might still be helpful...

\begin{itemize}
\item
State succinctly and clearly the problem you are trying to solve
and the progress that will be made toward doing so if the proposed
observations are successful.  If the review panel members have to work hard
even to understand what you want to do, they are unlikely to be
sympathetic to your proposal.

\item
Explain clearly why the project is important and how it
relates to the broad context and important issues in your field.
Many proposals focus too tightly on a specific observational
goal (e.g. ``measure the velocity dispersion of this cluster of galaxies'')
without explaining why it is important or how it relates to a
significant question about the Universe.

\item
Be specific.  If your observations will ``constrain theoretical
models,'' then discuss what will be constrained and why those
constraints matter.  Make sure the review panel understands exactly why
the observations you propose will make a difference in your field,
and exactly how the observations will refine or
require changes in the theory.

\item
Keep it simple.  Try to focus on the central idea of your proposal.
Complex arguments are hard to explain and hard for the panel members to follow.
Distracting tangential arguments obscure the theme of your proposal.

\item
Include a figure to help explain what you want to do.  Sample
data or model predictions shown in a figure often help clarify
complex arguments for the panel members.  A sample figure is included
below with this proposal.

\item
Keep it short.  Never exceed a page for the text of the scientific
justification, and never reduce the font size.  It may even help to
be a little under a page, and increase the font size a little!
Organize your presentation with paragraphs, headings, and bullets
so it is easy to read.

\item
Include and check references as appropriate.

\item
Print out the proposal to be sure your LaTeX is correct.
Proofread it.  Make sure the proposal is correct scientifically,
technically, and grammatically.  Run a spellchecker.

\end{itemize}

Finally, when an opportunity arises, volunteer to serve on
a TAC or review panel.  The experience is a great help in
learning how to write a good scientific justification.

\begin{references}
\reference Bell, D., Biemesderfer, C. D., Barnes, J., \& Massey, P. 1996, in
Astronomical Data Analysis Software and Systems V, A.S.P.\ Conf.\ Ser.,
Vol.\ 101, eds. G. H. Jacoby \& J. Barnes (San Francisco: ASP), 451
\end{references}

\begin{figure}[hbt]
\epsscale{0.85}
\plotone{figure1a.eps}
\caption{This sample figure shows how quickly electronic proposals
for telescope time replaced paper ones.}
\end{figure}

\clearpage


% EXPERIMENTAL DESIGN
%
% This section should consist of text only (no figures).
% There is a limit of one page of printed text.

% Describe your overall observational program.  How will these 
% observations contribute toward the accomplishment of the goals 
% outlined in the science justification?  If you've requested 
% long-term status, justify why this is necessary for successful 
% completion of the science.
%
% NOTE: If this project involves observations on other telescopes as
% well, list all telescopes on which you have applied for or been 
% granted time for observations related to this project.  For each, 
% indicate the nature of the observations, and describe the importance 
% of the observations proposed here in the context of the entire 
% program.

\expdesign
The review panel looks to this section to find out about the overall
strategy of your observational program.  Why do you need the telescopes
and instruments you request? How are your targets selected?
Why do you need spectroscopy or imaging, and what measurements will
you make from the data?  Why is your approach to be preferred over
some other approach, what must the minimum sample size be to achieve
your scientific goals (and why), and why are your
observations likely to be better than previous work in the field?

We also ask you to provide information about observations to be made
at other observatories as part of this program.  This information could
be as simple as the following:

{\bf
\begin{tabular}{ll}
Lick 1.0-m &  Broad-band Photometry \\
Keck  & Low dispersion spectroscopy of the faintest galaxies \\
\end{tabular}

Time is requested through NOAO to identify target objects on
Keck and Gemini.  Gemini time is needed for IR images and spectra
of the faintest program objects.
}



% LONG-TERM DETAILS
%
% If you are requesting long-term status for this proposal briefly 
% state the requirements for telescope time (telescope, instrument, 
% number of nights) needed in subsequent semesters to complete this 
% project in % \longtermdetails (be sure to uncomment the 
% \longtermdetails line below).
%
% Long-term status may NOT be requested for Gemini runs during this
% observing semester.
%
% If this is a long-term request you MAY ALSO NEED to modify the
% \proposaltype keyword at the top of this form changing "Standard"
% to "Longterm" (where is says "Please do not modify or delete this 
% line!").  It is this keyword that will flag this proposal as a 
% long-term status request, regardless of what may be entered here 
% in \longtermdetails!
%
% If this is not a long-term status request then please ignore this
% section.
%
% Long-term status is not applicable to CNTAC proposals.
%
\longtermdetails
We request longterm status on the CT-1.5m and Cassegrain spectrograph
with 5 nights each semester for 3 additional semesters.

% PAST USE
%
% How effectively have you used the facilities available through NOAO
% in the past?
% List allocations of telescope time on facilities available through 
% NOAO to the Principal Investigator during the past 2 years, together 
% with the current status of the data (cite publications where 
% appropriate).  Mark any allocations of time related to the current 
% proposal with a \relatedwork{} command.

\thepast
In this section, you should simply list the PI's recent telescope allocations
at any facilities available through NOAO, what's been done with the data, 
and what publications have resulted or are in progress.  It is, of course, 
helpful if the panel members can see that you do publish the results from 
previous observing runs in a timely way.
This is also a good place to highlight important results from previous 
runs with a sentence or two.

% ***FOR PROPOSALS REQUESTING TIME AT CTIO***
%
% Why CTIO?  Explain why access to the southern hemisphere is needed 
% to achieve your scientific goals.

\whyctio
This question need only be answered on proposals for telescope time at CTIO.
Explain why access to the southern hemisphere is
important to your program.  If you have access to other southern hemisphere
telescopes (e.g. the AAT or ESO) you may want to explain why you need access
to CTIO, as well.

%%%%%%%%%%%%%%%%%%%%%%%%%%%%%%%%%%%%%%%%%%%%%%%%%%%%%%%%%%%%%%%%%%%%%%

% OBSERVING RUN DETAILS - REQUIRED FOR EACH OBSERVING RUN REQUESTED
%
% For each run requested earlier in the \begin{obsruns}-\end{obsruns}
% sections of this proposal form, further run information must be
% specified.  Enter this block of information for each non-Gemini run.

% The \runid field must contain the run number plus 
% telescope/instrument-detector information as it appears for each 
% run in the obsruns sections. For example, 
% \runid{1}{KP-4m/ECHUV + T2KB}.

\runid{1}{HET/LRS}

% Describe the observations to be made during this observing run in
% the \technicaldescription section. Justify the specific telescope,
% the number of nights, the instrument, and the lunar phase requested.
% List objects, coordinates, and magnitudes (or surface brightness, if
% appropriate) using a LaTeX-coded table in this section or optionally
% enter the target information using the Target Tables described at the
% end of this template.  Target Tables of objects are required for 
% queue or service runs.
%
% Exposure time calculators (ETCs) for some optical and IR instruments
% in use at CTIO and at KPNO are available to assist you with the 
% preparation of your proposal.  See the Web pages:
%    Imaging ETC      - http://www.noao.edu/gateway/ccdtime/
%    Spectroscopy ETC - http://www.noao.edu/gateway/spectime/

\technicaldescription
HET observations will be taken in queue mode, and target tables
are required so that the TAC can understand the scope and complexity
of your program.  You may use the Web target table facility to
prepare your target list, or present them here in a LaTeX tabular
environment. If your program is approved, however, you will need
to prepare HET Phase II files with the actual details of your
observations for the HET staff.  The information in this section
is provided for the TAC rather than for the HET queue observers.


% Several instrument configuration parameters are requested.  Fill
% these in as appropriate for each run.  Required fields for
% WIYN/Hydra and WIYN/DensePak are noted with a **.
%
% \begin{configuration}
% \filters{}            % List here any filters that you plan to use.
% \grating{}            % List any gratings/grisms need with this run.**
% \order{}              % Specify any grating order(s).**
% \crossdisperser{}     % List cross disperser, if needed.
% \slit{}               % Enter slit widths you plan to use.
% \multislit{}          % yes or no only
% \wstart{}             % Starting wavelength of wavelength range.**
% \wend{}               % Ending wavelength of wavelength range.**
% \cable{}              % For CTIO/Hydra: enter 2.0".  For WIYN: enter
%                         red, blue, or densepak.**
% \corrector{}          % Enter red or blue for KP-coude, CAM5.
% \collimator{}         % Enter collimator needed.
% \adc{}                % If user selectable, enter yes or no only.
% \end{configuration}
%
% Details about these fields are available in the online help for the
% Web form at http://www.noao.edu/noaoprop/help/standard.html#iconfig

\begin{configuration}
\filters{}
\grating{}
\order{}
\crossdisperser{}       
\slit{2"}
\multislit{yes}            
\wstart{}
\wend{}
\cable{}
\corrector{}            
\collimator{}             
\adc{}
\end{configuration}


% Use \specialrequest to describe briefly any special or non-standard
% usage of NOAO or MMT instrumentation.   This will be ignored for HET
% runs if included.

%\specialrequest


% If you plan to submit any Target Tables for this run they must be
% entered here.  Target Tables are required for all queue/service runs.

% TARGET TABLES
%
% Target Tables are required for all queue or service
% runs and are optional for classical observing.  Target
% Tables, if included, are associated with each run in the observing
% run details section of the proposal form and must follow the 
% \specialrequest in each section.
%
% WIYN Mini-Mosaic, WIYN Bench Spectrograph (Hydra and DensePak), and 
% YALO tables require that all the fields in the target table be
% specified.  For other tables specific fields may be deleted EXCEPT 
% for the \obscomment command as mentioned below.
%
% WIYN and YALO investigators should supply good coordinates
% with the \ra and \dec commands since these values may be used to
% observe your fields.
%
% Note that for iterative targets, only the parameters that need
% to be changed have to be specified.  Once a parameter is specified in
% a targettable environment, it is retained until explicitly changed.
%
% The \obscomment command is REQUIRED for each target entry and
% must be the last item; this command forces each target line to be
% printed.  If no comment is needed, leave the argument blank.
%
% The \begin{targettable}{} command for each table must contain the
% telescope/instrument-detector information for that particular run,
% i.e,. \begin{targettable}{KP-4m/ECHUV + T2KB}.
%
% HINTS: Long tables do not break across pages. If it is necessary to
% continue a table across a page you must start a new table.  Use
% /clearpage before the \begin{targettable} command for the new table.
%
%\begin{targettable}{}
%\objid{}           % WIYN: specify a 3-digit number for each target.
%\object{}          % 20 characters maximum
%\ra{}              % e.g., xx:xx:xx.x
%\dec{}             % e.g., +-xx:xx:xx.x
%\epoch{}           % e.g., 1950.3
%\magnitude{}
%\filter{}
%\exptime{}         % WIYN: in seconds PER EXPOSURE
%\nexposures{}      % Number of exposures
%\moondays{}        % Days from new moon, use a number 0-14
%\skycond{}         % "spec" or "phot"
%\seeing{}          % max allowable PSF FWHM (arcsecs)
%\obscomment{}      % 20 characters maximum - REQUIRED COMMAND
%  - repeat target entry parameters as needed to complete Target Table -
%\end{targettable}

\begin{targettable}{Run 1: HET/LRS}

\objid{777}
\object{NGC 7078}
\ra{21:30.1}
\dec{12:10}
\epoch{2000.0}
\magnitude{18.6}
\filter{GG-475}
\exptime{1000}
\nexposures{5}
\moondays{4}
\skycond{phot}
\seeing{1.5-2.0}
\obscomment{globular cluster}

\objid{778}
\obscomment{same cluster}

\objid{779}
\obscomment{}

\end{targettable}


%%%%%%%%%%%%%%%%%%%%%%%%%%%%%%%%%%%%%%%%%%%%%%%%%%%%%%%%%%%%%%%%%%%%%%%%%%%%


\runid{2}{MMT/BCHAN}

\technicaldescription
For this run on the MMT Blue Channel Spectrograph, you may want to explain
why you have selected this wavelength region, what velocity resolution
and S/N ratio you need and why, and mention any special procedures for
calibration and data reduction that you plan to use.  Again, justify
the amount of time requested in detail.  You may also need to justify
the specific restriction on days from new moon.

In the instrument configuration table below, you should indicate which
grating and order you will use, whether you will use cross dispersed
mode with the echellette grating, which slit you expect to use, and
the starting and ending wavelengths.  Note that a multi-slit mode
is not available at this time.

\begin{configuration}
\filters{}
\grating{300g/mm}
\order{1}
\crossdisperser{no}       
\slit{1"}
\multislit{}            
\wstart{4000}
\wend{9000}
\cable{}
\corrector{}            
\collimator{}             
\adc{}
\end{configuration}

\specialrequest



%%%%%%%%%%%%%%%%%%%%%%%%%%%%%%%%%%%%%%%%%%%%%%%%%%%%%%%%%%%%%%%%%%%%%%%%%%%%


\runid{3}{KP-4m/RCSPM + T2KB}

\technicaldescription
For this run on KPNO's Mayall 4-m telescope with the RC spectrograph
plus multislits, you might want to explain why you have chosen the 
indicated grating and wavelength region, what your S/N ratio and resolution
requirements are, how many multi-slit plates you will need, where the
coordinates will come from, what you estimate the exposure times will
be, and why you need the amount of time requested to complete the program.

\begin{configuration}
\filters{GG-495}
\grating{BL-450}
\order{2}
\crossdisperser{}       
\slit{}
\multislit{yes}            
\wstart{5000}
\wend{6000}
\cable{}
\corrector{}            
\collimator{}             
\adc{}
\end{configuration}

\specialrequest

\begin{targettable}{Run 3: KP-4m/RCSPM + T2KB}

\object{NGC 6205}
\ra{19:45:12.8}
\dec{13:30:40.3}
\epoch{1950.0}
\magnitude{16-19}
\moondays{12}
\seeing{0.7}
\obscomment{}

\object{NGC 6205}
\ra{19:45:12.8}
\dec{13:30:40.3}
\epoch{1950.0}
\moondays{12}
\seeing{0.7}
\obscomment{}

\object{NGC 6205}
\ra{19:45:12.8}
\dec{13:30:40.3}
\epoch{1950.0}
\magnitude{16-19}
\moondays{12}
\seeing{0.7}
\obscomment{}

\end{targettable}



%%%%%%%%%%%%%%%%%%%%%%%%%%%%%%%%%%%%%%%%%%%%%%%%%%%%%%%%%%%%%%%%%%%%%%%%%%%%


\runid{4}{WIYN/HYDRR + T2KC}

\technicaldescription
For WIYN queue programs (WIYN-2hr and WIYN-SYN), be sure to provide enough 
technical details that
the WIYN queue observers can carry out your program.  See 
http://www.noao.edu/wiyn/obsprog/ for details.

For this run with the red camera on Hydra, you may want to explain
why you select a particular grating/wavelength region/resolution.
For WIYN queue observations, you should describe in detail a
per-exposure figure of merit that the queue observers can use
to determine if an individual observation is adequate for your
needs.  You should describe any special calibration requirements
(e.g. do you need a daylight sky spectrum?  Twilight flats?).

\begin{configuration}
\filters{}
\grating{316 l/mm}
\order{1}
\crossdisperser{}       
\slit{}
\multislit{}            
\wstart{4000}
\wend{8000}
\cable{red}
\corrector{}            
\collimator{}             
\adc{}
\end{configuration}

\specialrequest

\begin{targettable}{Run 4: WIYN/HYDRR + T2KC}

\objid{777}
\object{NGC 6205}
\ra{19:45:12.8}
\dec{13:30:40.3}
\epoch{1950.0}
\magnitude{16-19}
\filter{GC-495}
\exptime{1000}
\nexposures{5}
\moondays{4}
\skycond{phot}
\seeing{0.7}
\obscomment{globular cluster}

\objid{778}
\obscomment{same cluster}

\objid{779}
\obscomment{}

\end{targettable}



%%%%%%%%%%%%%%%%%%%%%%%%%%%%%%%%%%%%%%%%%%%%%%%%%%%%%%%%%%%%%%%%%%%%%%%%%%%%


\runid{5}{CT-1.5m/CSPEC + L1K}

\technicaldescription
For this run with CT-1.5m Cassegrain spectrograph,
you may want to explain why you have selected this
particular wavelength region, what velocity resolution you need,
and why, and any special procedures for calibration and data
reductions that you plan to use.  Again, justify the amount
of time requested in detail.  You may also need to justify
the specific restriction on days from new moon.

\begin{configuration}
\filters{}
\grating{9}
\order{}
\crossdisperser{}       
\slit{}
\multislit{}            
\wstart{3600}
\wend{6500}
\cable{}
\corrector{}            
\collimator{}             
\adc{}
\end{configuration}

\specialrequest
If you need non-sidereal tracking, for example, this might be a good
place to note that fact.



%%%%%%%%%%%%%%%%%%%%%%%%%%%%%%%%%%%%%%%%%%%%%%%%%%%%%%%%%%%%%%%%%%%%%%%%%

% Detailed information as noted below must be provided for all runs up to
% SIX runs.  Since Gemini runs must be specified through the Web form this
% section only applies to non-Gemini runs - use the Web form to generate
% detailed information and target tables for Gemini runs.  Target Tables
% are required for WIYN-2hr, WIYN-SYN, and YALO runs but are
% optional for other telescopes and instruments.
%
%\runid{}{}
%\technicaldescription
%\begin{configuration}
%\filters{}
%\grating{}
%\order{}
%\crossdisperser{}
%\slit{}
%\multislit{}
%\wstart{}
%\wend{}
%\cable{}
%\corrector{}
%\collimator{}
%\adc{}
%\end{configuration}
%\specialrequest    % remove and not use for HET runs
%\begin{targettable}{}
%\objid{}           % WIYN: specify a 3-digit number for each target
%\object{}          % 20 characters maximum
%\ra{}              % e.g., xx:xx:xx.x
%\dec{}             % e.g., +-xx:xx:xx.x
%\epoch{}           % e.g., 1950.3
%\magnitude{}
%\filter{}
%\exptime{}         % WIYN: in seconds PER EXPOSURE
%\nexposures{}      % Number of exposures
%\moondays{}        % Days from new moon, use a number 0-14
%\skycond{}         % "spec" or "phot"
%\seeing{}          % max allowable PSF FWHM (arcsecs)
%\obscomment{}      % 20 characters maximum - REQUIRED COMMAND
%  - repeat target entry parameters as needed to complete Target Table -
%\end{targettable}

%%%%%%%%%%%%%%%%%%%%%%%%%%%%%%%%%%%%%%%%%%%%%%%%%%%%%%%%%%%%%%%%%%%%%%%%

% Please do not modify or delete this line.
\end{document}

% Send the completed proposal file to noaoprop-submit@noao.edu.
%
% Figures should be submitted to noaoprop-submit@noao.edu AFTER
% the proposal identification number has been returned to you
% via email; this acknowledgement message will contain instructions
% on what to do.  If you submit figures without supplying
% the proper identifying information (per the email explanation), the
% figures stand a reasonable chance of being lost.  You will receive
% an acknowledgement message for each properly submitted figure.
%
% Expect these acknowledgement messages to be returned rather quickly.
% If you have not received email from us within 10-15 minutes of
% your submission contact us at noaoprop-help@noao.edu for assistance.
% And as a final reminder you may track your proposal processing by
% the proposal ID number at the following web page:
%         http://www.noao.edu/cgi-bin/noaoprop/propstatus
%
% Thank you for your interest in NOAO.  Contact us at
% noaoprop-help@noao.edu if you have any suggestions or comments.
