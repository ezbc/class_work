%-------------------------------------------------------------------------------
%	PACKAGES AND OTHER DOCUMENT CONFIGURATIONS
%-------------------------------------------------------------------------------

\documentclass[11pt]{article}

% Packages
% Packages

% \usepackage{fancyhdr} % Required for custom headers
% \usepackage{lastpage} % Required to determine the last page for the footer
% \usepackage{extramarks} % Required for headers and footers
% \usepackage[usenames,dvipsnames]{color} % Required for custom colors
\usepackage{graphicx} % Required to insert images
% \usepackage{listings} % Required for insertion of code
% \usepackage{courier} % Required for the courier font
% \usepackage{dsfont} % For special math characters
% \usepackage{verbatim}

%\usepackage{amsmath, amssymb, bm} % For matrix notation
\usepackage[english]{babel}
\usepackage[paperwidth=8.5in,paperheight=11in,margin=1.0in]{geometry}
\usepackage{listings}
\usepackage{hyperref}
%\usepackage[cmex10]{amsmath, bm}
\usepackage{amsmath, bm}
\usepackage{blkarray}








% formatting
\pdfcompresslevel0

% ==============================================================================
% PYTHON
% ==============================================================================
\usepackage[utf8]{inputenc}

% Default fixed font does not support bold face
\DeclareFixedFont{\ttb}{T1}{txtt}{bx}{n}{12} % for bold
\DeclareFixedFont{\ttm}{T1}{txtt}{m}{n}{12}  % for normal

% Custom colors
\usepackage{color}
\definecolor{deepblue}{rgb}{0,0,0.5}
\definecolor{deepred}{rgb}{0.6,0,0}
\definecolor{deepgreen}{rgb}{0,0.5,0}

\usepackage{listings}

% Python style for highlighting
\newcommand\pythonstyle{\lstset{
language=Python,
basicstyle=\ttm,
otherkeywords={self},             % Add keywords here
keywordstyle=\ttb\color{deepblue},
emph={MyClass,__init__},          % Custom highlighting
emphstyle=\ttb\color{deepred},    % Custom highlighting style
stringstyle=\color{deepgreen},
frame=tb,                         % Any extra options here
showstringspaces=false,            % 
breaklines=true
}}


% Python environment
\lstnewenvironment{python}[1][]
{\pythonstyle\lstset{#1}
}
{}

% Python for external files
\newcommand\pythonexternal[2][]{{
\pythonstyle\lstinputlisting[#1]{#2}}}

% Python for inline
\newcommand\pythoninline[1]{{\pythonstyle\lstinline!#1!}}
% ==============================================================================
% ==============================================================================

% Margins
\topmargin=-0.45in
\evensidemargin=0in
\oddsidemargin=0in
\textwidth=6.5in
\textheight=9.0in
\headsep=0.25in

\linespread{1.1} % Line spacing

% Set up the header and footer
\pagestyle{fancy}
\lhead{\hmwkAuthorName} % Top left header
\chead{\hmwkClass\ (\hmwkClassInstructor\ \hmwkClassTime): \hmwkTitle} % Top center head
\rhead{\firstxmark} % Top right header
\lfoot{\lastxmark} % Bottom left footer
\cfoot{} % Bottom center footer
\rfoot{Page\ \thepage\ of\ \protect\pageref{LastPage}} % Bottom right footer
\renewcommand\headrulewidth{0.4pt} % Size of the header rule
\renewcommand\footrulewidth{0.4pt} % Size of the footer rule

\setlength\parindent{0pt} % Removes all indentation from paragraphs

%----------------------------------------------------------------------------------------
%	DOCUMENT STRUCTURE COMMANDS
%	Skip this unless you know what you're doing
%----------------------------------------------------------------------------------------

% Header and footer for when a page split occurs within a problem environment
\newcommand{\enterProblemHeader}[1]{\nobreak\extramarks{#1}{#1 continued on next page\ldots}\nobreak\nobreak\extramarks{#1 (continued)}{#1 continued on next page\ldots}\nobreak}

% Header and footer for when a page split occurs between problem environments
\newcommand{\exitProblemHeader}[1]{\nobreak\extramarks{#1 (continued)}{#1 continued on next page\ldots}\nobreak\nobreak\extramarks{#1}{}\nobreak}

\setcounter{secnumdepth}{0} % Removes default section numbers
\newcounter{homeworkProblemCounter} % Creates a counter to keep track of the number of problems

\newcommand{\homeworkProblemName}{}
\newenvironment{homeworkProblem}[1][Problem \arabic{homeworkProblemCounter}]{ % Makes a new environment called homeworkProblem which takes 1 argument (custom name) but the default is "Problem #"
\stepcounter{homeworkProblemCounter} % Increase counter for number of problems
\renewcommand{\homeworkProblemName}{#1} % Assign \homeworkProblemName the name of the problem
\section{\homeworkProblemName} % Make a section in the document with the custom problem count
\enterProblemHeader{\homeworkProblemName} % Header and footer within the environment
}{\exitProblemHeader{\homeworkProblemName} % Header and footer after the environment
}

% Defines the problem answer command with the content as the only argument
\newcommand{\problemAnswer}[1]{\noindent\framebox[\columnwidth, resolution=600][c]{\begin{minipage}{0.98\columnwidth, resolution=600}#1\end{minipage}}}
% Makes the box around the problem answer and puts the content inside }

\newcommand{\homeworkSectionName}{}
\newenvironment{homeworkSection}[1]{ % New environment for sections within homework problems, takes 1 argument - the name of the section
\renewcommand{\homeworkSectionName}{#1} % Assign \homeworkSectionName to the name of the section from the environment argument
\subsection{\homeworkSectionName} % Make a subsection with the custom name of the subsection
\enterProblemHeader{\homeworkProblemName\ [\homeworkSectionName]} % Header and footer within the environment
}{
\enterProblemHeader{\homeworkProblemName} % Header and footer after the environment
}



\bibliographystyle{apj}
\usepackage{float} % for figure placement
\usepackage{natbib} % for bibtex referencing
\usepackage{cite} % cite with bibtex
\usepackage{aas_macros} % for journal referencing
\usepackage{hyperref} % for including urls

%-------------------------------------------------------------------------------
%	NAME AND CLASS SECTION
%-------------------------------------------------------------------------------

\newcommand{\hmwkTitle}{HW 11} % Assignment title
\newcommand{\hmwkDueDate}{Tuesday, Oct. 28} % Due date
\newcommand{\hmwkClass}{Stat 860} % Course/class
\newcommand{\hmwkClassTime}{4:00 PM} % Class/lecture time
\newcommand{\hmwkClassInstructor}{Grace Wahba} % Teacher/lecturer
\newcommand{\hmwkAuthorName}{Elijah Bernstein-Cooper} % Your name

%-------------------------------------------------------------------------------
%	TITLE PAGE
%-------------------------------------------------------------------------------

\title{\vspace{0in}
    \textmd{\textbf{\hmwkClass:\ \hmwkTitle}}\\
    \normalsize\vspace{0.1in}\small{Due\ on\ \hmwkDueDate}\\
    \vspace{0.1in}\large{\textit{\hmwkClassInstructor\ \hmwkClassTime}}
    \vspace{0.2in}}

\author{\textbf{Elijah Bernstein-Cooper}}
\date{\today} % Insert date here if you want it to appear below your name

%-------------------------------------------------------------------------------

\begin{document}

\maketitle
%\newpage

%===============================================================================
%-------------------------------------------------------------------------------
%	PROBLEM 1
%-------------------------------------------------------------------------------
\begin{homeworkProblem}

    Abstract for \citet{lindner:vera-ciro:murray:stanimirovic:2014}:

    \begin{quote}

        We present a new algorithm, named Autonomous Gaussian Decomposition
        (AGD), for automatically decomposing spectra into Gaussian components.
        AGD uses derivative spectroscopy and machine learning to provide
        optimized guesses for the number of Gaussian components in the data,
        and also their locations, widths, and amplitudes. We test AGD and find
        that it produces results comparable to human-derived solutions on 21cm
        absorption spectra from the 21cm SPectral line Observations of Neutral
        Gas with the EVLA (21-SPONGE) survey. We use AGD with Monte Carlo
        methods to derive the HI line completeness as a function of peak
        optical depth and velocity width for the 21-SPONGE data, and also show
        that the results of AGD are stable against varying observational noise
        intensity. The autonomy and computational efficiency of the method over
        traditional manual Gaussian fits allow for truly unbiased comparisons
        between observations and simulations, and for the ability to scale up
        and interpret the very large data volumes from the upcoming Square
        Kilometer Array and pathfinder telescopes.
        
    \end{quote}

    URL: \url{http://arxiv.org/abs/1409.2840} \\

    \clearpage
    Abstract for \citet{yeow:azali:ow:wong:2005}:
    
    \begin{quote}
        
        The problem of differentiating spectral data to yield the third and
        fourth derivatives is converted into one of solving an integral
        equation of the first kind. This equation is solved by Tikhonov
        regularization. The method of General Cross Validation is used to guide
        the choice of the regularization parameter that keeps noise
        amplification under control. The performance of this route to third and
        fourth derivative spectra is demonstrated by applying it to a number of
        published spectra. A computational problem associated with General
        Cross Validation has been identified.

    \end{quote}

    URL: \url{http://www.hindawi.com/journals/isrn/2011/164564/abs/} \\

    Abstract for \citet{chartrand:2011}:

    \begin{quote}

        We consider the problem of differentiating a function specified by
        noisy data. Regularizing the differentiation process avoids the noise
        amplification of finite-difference methods. We use total-variation
        regularization, which allows for discontinuous solutions. The resulting
        simple algorithm accurately differentiates noisy functions, including
        those which have a discontinuous derivative.

    \end{quote}

    URL:
    \url{http://www.sciencedirect.com/science/article/pii/S0039914005003267} \\

\end{homeworkProblem}
%===============================================================================

    \bibliography{refs}

\end{document}

