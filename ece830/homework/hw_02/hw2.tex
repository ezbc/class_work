%-------------------------------------------------------------------------------
%	PACKAGES AND OTHER DOCUMENT CONFIGURATIONS
%-------------------------------------------------------------------------------

\documentclass{article}

% Packages
% Packages

% \usepackage{fancyhdr} % Required for custom headers
% \usepackage{lastpage} % Required to determine the last page for the footer
% \usepackage{extramarks} % Required for headers and footers
% \usepackage[usenames,dvipsnames]{color} % Required for custom colors
\usepackage{graphicx} % Required to insert images
% \usepackage{listings} % Required for insertion of code
% \usepackage{courier} % Required for the courier font
% \usepackage{dsfont} % For special math characters
% \usepackage{verbatim}

%\usepackage{amsmath, amssymb, bm} % For matrix notation
\usepackage[english]{babel}
\usepackage[paperwidth=8.5in,paperheight=11in,margin=1.0in]{geometry}
\usepackage{listings}
\usepackage{hyperref}
%\usepackage[cmex10]{amsmath, bm}
\usepackage{amsmath, bm}
\usepackage{blkarray}








% formatting
\pdfcompresslevel0

% ==============================================================================
% PYTHON
% ==============================================================================
\usepackage[utf8]{inputenc}

% Default fixed font does not support bold face
\DeclareFixedFont{\ttb}{T1}{txtt}{bx}{n}{12} % for bold
\DeclareFixedFont{\ttm}{T1}{txtt}{m}{n}{12}  % for normal

% Custom colors
\usepackage{color}
\definecolor{deepblue}{rgb}{0,0,0.5}
\definecolor{deepred}{rgb}{0.6,0,0}
\definecolor{deepgreen}{rgb}{0,0.5,0}

\usepackage{listings}

% Python style for highlighting
\newcommand\pythonstyle{\lstset{
language=Python,
basicstyle=\ttm,
otherkeywords={self},             % Add keywords here
keywordstyle=\ttb\color{deepblue},
emph={MyClass,__init__},          % Custom highlighting
emphstyle=\ttb\color{deepred},    % Custom highlighting style
stringstyle=\color{deepgreen},
frame=tb,                         % Any extra options here
showstringspaces=false,            % 
breaklines=true
}}


% Python environment
\lstnewenvironment{python}[1][]
{\pythonstyle\lstset{#1}
}
{}

% Python for external files
\newcommand\pythonexternal[2][]{{
\pythonstyle\lstinputlisting[#1]{#2}}}

% Python for inline
\newcommand\pythoninline[1]{{\pythonstyle\lstinline!#1!}}
% ==============================================================================
% ==============================================================================

% Margins
\topmargin=-0.45in
\evensidemargin=0in
\oddsidemargin=0in
\textwidth=6.5in
\textheight=9.0in
\headsep=0.25in

\linespread{1.1} % Line spacing

% Set up the header and footer
\pagestyle{fancy}
\lhead{\hmwkAuthorName} % Top left header
\chead{\hmwkClass\ (\hmwkClassInstructor\ \hmwkClassTime): \hmwkTitle} % Top center head
\rhead{\firstxmark} % Top right header
\lfoot{\lastxmark} % Bottom left footer
\cfoot{} % Bottom center footer
\rfoot{Page\ \thepage\ of\ \protect\pageref{LastPage}} % Bottom right footer
\renewcommand\headrulewidth{0.4pt} % Size of the header rule
\renewcommand\footrulewidth{0.4pt} % Size of the footer rule

\setlength\parindent{0pt} % Removes all indentation from paragraphs

%----------------------------------------------------------------------------------------
%	DOCUMENT STRUCTURE COMMANDS
%	Skip this unless you know what you're doing
%----------------------------------------------------------------------------------------

% Header and footer for when a page split occurs within a problem environment
\newcommand{\enterProblemHeader}[1]{\nobreak\extramarks{#1}{#1 continued on next page\ldots}\nobreak\nobreak\extramarks{#1 (continued)}{#1 continued on next page\ldots}\nobreak}

% Header and footer for when a page split occurs between problem environments
\newcommand{\exitProblemHeader}[1]{\nobreak\extramarks{#1 (continued)}{#1 continued on next page\ldots}\nobreak\nobreak\extramarks{#1}{}\nobreak}

\setcounter{secnumdepth}{0} % Removes default section numbers
\newcounter{homeworkProblemCounter} % Creates a counter to keep track of the number of problems

\newcommand{\homeworkProblemName}{}
\newenvironment{homeworkProblem}[1][Problem \arabic{homeworkProblemCounter}]{ % Makes a new environment called homeworkProblem which takes 1 argument (custom name) but the default is "Problem #"
\stepcounter{homeworkProblemCounter} % Increase counter for number of problems
\renewcommand{\homeworkProblemName}{#1} % Assign \homeworkProblemName the name of the problem
\section{\homeworkProblemName} % Make a section in the document with the custom problem count
\enterProblemHeader{\homeworkProblemName} % Header and footer within the environment
}{\exitProblemHeader{\homeworkProblemName} % Header and footer after the environment
}

% Defines the problem answer command with the content as the only argument
\newcommand{\problemAnswer}[1]{\noindent\framebox[\columnwidth, resolution=600][c]{\begin{minipage}{0.98\columnwidth, resolution=600}#1\end{minipage}}}
% Makes the box around the problem answer and puts the content inside }

\newcommand{\homeworkSectionName}{}
\newenvironment{homeworkSection}[1]{ % New environment for sections within homework problems, takes 1 argument - the name of the section
\renewcommand{\homeworkSectionName}{#1} % Assign \homeworkSectionName to the name of the section from the environment argument
\subsection{\homeworkSectionName} % Make a subsection with the custom name of the subsection
\enterProblemHeader{\homeworkProblemName\ [\homeworkSectionName]} % Header and footer within the environment
}{
\enterProblemHeader{\homeworkProblemName} % Header and footer after the environment
}



%-------------------------------------------------------------------------------
%	NAME AND CLASS SECTION
%-------------------------------------------------------------------------------

\newcommand{\hmwkTitle}{Homework 2} % Assignment title
\newcommand{\hmwkDueDate}{Monday, February 9} % Due date
\newcommand{\hmwkClass}{ECE 830} % Course/class
\newcommand{\hmwkClassTime}{11:00 am} % Class/lecture time
\newcommand{\hmwkClassInstructor}{Robert Nowak} % Teacher/lecturer
\newcommand{\hmwkAuthorName}{Elijah Bernstein-Cooper} % Your name

%-------------------------------------------------------------------------------
%	TITLE PAGE
%-------------------------------------------------------------------------------

\title{\vspace{0in}
    \textmd{\textbf{\hmwkClass:\ \hmwkTitle}}\\
    \normalsize\vspace{0.1in}\small{Due\ on\ \hmwkDueDate}\\
    \vspace{0.1in}\large{\textit{\hmwkClassInstructor\ \hmwkClassTime}}
    \vspace{0.5in}}

\author{\textbf{Elijah Bernstein-Cooper}}
\date{\today} % Insert date here if you want it to appear below your name

%-------------------------------------------------------------------------------

\begin{document}

\maketitle
%\newpage

%===============================================================================
%-------------------------------------------------------------------------------
%	PROBLEM 1
%-------------------------------------------------------------------------------
\begin{homeworkProblem}

    \begin{homeworkSection}{1a}

        An unbiased estimator of $\lambda$ is the expectation of $X_i$. The MSE
        of the expectation is $\frac{\lambda}{n}$.
    
    \end{homeworkSection}

    \begin{homeworkSection}{1b}
        

        
    \end{homeworkSection}

\end{homeworkProblem}
%===============================================================================

%===============================================================================
%-------------------------------------------------------------------------------
%	PROBLEM 2
%-------------------------------------------------------------------------------
\begin{homeworkProblem}

    \begin{homeworkSection}{2a}

        The covariance $\Sigma_{XX}$ of $\bm{X}$ is given by

        \begin{eqnarray*}
            \Sigma_{XX} & = & \mathbb{E}[(\bm{H\theta} -
                                          \mathbb{E}[\bm{H\theta}])
                                         (\bm{H\theta} -
                                          \mathbb{E}[\bm{H\theta}])^T] \\
            \Sigma_{XX} & = & \mathbb{E}[(\bm{H\theta} -
                                  \mathbb{E}[\bm{H}]\mathbb{E}[\bm{\theta}])
                                         (\bm{H\theta} -
                               \mathbb{E}[\bm{H}]\mathbb{E}[\bm{\theta}])^T]\\
        \end{eqnarray*}

        \noindent where $\mathbb{E}[\theta] = 0$ hence

        \begin{eqnarray*}
            \Sigma_{XX} & = & \mathbb{E}[(\bm{H\theta})(\bm{H\theta})^T] \\
            \Sigma_{XX} & = & \mathbb{E}[(\bm{\theta H H}^T \bm{\theta})] \\
            \Sigma_{XX} & = & \sigma_\theta^2 \bm{H H}^T. \\
        \end{eqnarray*}

        A similar analysis for $\Sigma_{YY}$ yields 

        \begin{eqnarray*}
            \Sigma_{YY} & = & \sigma_\theta^2 \bm{H H}^T + \sigma_W^2
            \bm{I}  \\
        \end{eqnarray*}

    \end{homeworkSection}

    \begin{homeworkSection}{2b}

        The eigenvectors of $\Sigma_{YY}$ are related to $\bm{H}$ by the
        following

        \begin{equation*}
            \Sigma_{YY} = \sum_{i=1}^n (\bm{u}_i^T (\sigma_\theta^2 \bm{HH}^T +
            \sigma_W^2 \bm{I})) \bm{u}_i
        \end{equation*}

        \noindent where the vectors $\bm{u}_i$ make up the columns of the
        matrix $\bm{U}$. The eigenvalues of $\Sigma_{YY}$ are then given by

        \begin{equation*}
            \Sigma_{YY} = \bm{UDU}^*
        \end{equation*}

        \noindent where $\bm{D}$ is a diagonal matrix whose diagonal entries
        are eigenvalues.

    \end{homeworkSection}

    \begin{homeworkSection}{2c}

        We perform an eigendecomposition of the covariance matrix for the
        example convolution of sinusoids using the \texttt{eig} function. We
        find that the number of eigenvalues above the standard deviation in the
        data, 0.5, is 5. See the code at the end of the homework.

    \end{homeworkSection}


\end{homeworkProblem}
%===============================================================================

%===============================================================================
%-------------------------------------------------------------------------------
%	PROBLEM 3
%-------------------------------------------------------------------------------
\begin{homeworkProblem}

    \begin{homeworkSection}{3a}

        \begin{eqnarray*}
            ||X - X_r|| &=& ||\sum_{i=1}^n (u_i^T X)u_i - 
                            \sum_{i=1}^r (u_i^T X)u_i|| \\
            ||X - X_r|| &=& ||\sum_{i=r+1}^n (u_i^T X)u_i|| \\
                            ||X - X_r|| &=& \sqrt{\sum_{j=1}^n
                            \left(\sum_{i=r+1}^n (u_{ij}^T
                        X_j)u_{ij}\right)^2} \\
        \end{eqnarray*}

        \noindent thus

        \begin{eqnarray*}
            \mathbb{E}[||X - X_r||^2] &=& \sum_{i=r+1}^n \lambda_i
        \end{eqnarray*}
        
    \end{homeworkSection}

    \begin{homeworkSection}{3b}

        The total variability in the mean vector can be represented as the sum
        of the eigenvalues. We thus count the number of eigenvalues needed to
        sum to 0.95\% the variability of the mean vector. We find that the
        first 97 eigenvectors are needed to account for 95\% of the variability
        about the mean vector.

    \end{homeworkSection}

\end{homeworkProblem}
\clearpage
%===============================================================================

\clearpage
{\huge Code:\\}

{\large \bf Problem 1} \\
\lstinputlisting{hw1_prob1.m} 

{\large \bf Problem 2c} \\
\lstinputlisting{hw1_prob2.m} 

{\large \bf Problem 3a} \\
\lstinputlisting{hw1_prob3.m} 

\end{document}

