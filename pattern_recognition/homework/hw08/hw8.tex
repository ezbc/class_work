%-------------------------------------------------------------------------------
%	PACKAGES AND OTHER DOCUMENT CONFIGURATIONS
%-------------------------------------------------------------------------------

\documentclass{article}

% Packages
% packages to load
\usepackage{amssymb} %
\usepackage[utf8]{inputenc}
%\usepackage[margin=1in]{geometry}
\usepackage{graphics,graphicx} % jpg and pdf figures
%\usepackage{sidecap}
\usepackage{amsmath}
\usepackage{verbatim} % block comments
\usepackage{multicol, caption} % multicolumn formating
\usepackage{lipsum} % needed for figures in multicol format
\usepackage{float} % for figure placement
\usepackage{natbib} % for bibtex referencing
\usepackage{cite} % cite with bibtex
\usepackage{aas_macros} % for journal referencing





% formatting
%%%%%%%%%%%%%%%%%%%%%%%%%%%%
%%%%%% Page dimensions %%%%%
%%%%%%  DO NOT CHANGE  %%%%%
%%%%%%%%%%%%%%%%%%%%%%%%%%%%

\textheight=247mm
\textwidth=180mm
\topmargin=-7mm
\oddsidemargin=-10mm
\evensidemargin=-10mm
\parindent 10pt

% Citation format
\bibliographystyle{apj}
\citestyle{aa}



%-------------------------------------------------------------------------------
%	NAME AND CLASS SECTION
%-------------------------------------------------------------------------------

\newcommand{\hmwkTitle}{Homework 8} % Assignment title
\newcommand{\hmwkDueDate}{Friday, Nov 14} % Due date
\newcommand{\hmwkClass}{ECE 532} % Course/class
\newcommand{\hmwkClassTime}{11:00 am} % Class/lecture time
\newcommand{\hmwkClassInstructor}{Robert Nowak} % Teacher/lecturer
\newcommand{\hmwkAuthorName}{Elijah Bernstein-Cooper} % Your name

%-------------------------------------------------------------------------------
%	TITLE PAGE
%-------------------------------------------------------------------------------

\title{\vspace{0in}
    \textmd{\textbf{\hmwkClass:\ \hmwkTitle}}\\
    \normalsize\vspace{0.1in}\small{Due\ on\ \hmwkDueDate}\\
    \vspace{0.1in}\large{\textit{\hmwkClassInstructor\ \hmwkClassTime}}
    \vspace{0.5in}}

\author{\textbf{Elijah Bernstein-Cooper}}
\date{\today} % Insert date here if you want it to appear below your name

%-------------------------------------------------------------------------------

\begin{document}

\maketitle
%\newpage

%===============================================================================
%-------------------------------------------------------------------------------
%	PROBLEM 1
%-------------------------------------------------------------------------------
\begin{homeworkProblem}
   
    We performed a brute force estimate of the dual optimization parameter
    $\alpha$ for hinge loss to classify the four people as basketball players
    or not basketball players. We optimized the following

    \begin{equation}
        {\rm min_{\alpha}} \sum_{i=1}^m \left(1 - b_i\sum_{j=1}^m\alpha_j 
        \bm{a_i}^T \bm{a}_j \right) + \lambda \sum_{i=1}^m \sum_{j=1}^m \alpha_i
        \alpha_j \bm{a}_i^T \bm{a}_j
    \end{equation}

    \noindent where $\alpha$ will be a 4$\times$1 vector of weights, or rather
    the support vector. We derived
  
    \begin{equation}
    \bm{a} = \left(\begin{matrix} -0.8 \\ -0.1 \\ 0 \\ 0.8 \\ 
    \end{matrix} \right)
    \end{equation}

    We wish to derive a best estimate for the weight vector $\bm{x}$ where
    $\bm{b} = \bm{Ax}$, with $\bm{b}$ being a binary vector, 1 for basketball
    player, -1 for not a basketball player, and $\bm{A}$ will be the heights of
    the people. In the dual hinge loss scenario

    \begin{equation}
        \bm{\hat{x}} = \sum_{j=1}^m \alpha_j \bm{a}_j
    \end{equation}

    Our hinge loss optimization produced an optimal classification rule of
    $\bm{\hat{x}} = 0.21$. Using least squares we find an $\bm{\hat{x}}$ of
    0.0076. The least squares classification predicts that all people will be
    basketball players, while the dual hinge loss SVM classification predicts
    the two tallest will be basketball players. See the code at the end of the
    homework.

\end{homeworkProblem}
\clearpage
%===============================================================================

%===============================================================================
%-------------------------------------------------------------------------------
%	PROBLEM 2 
%-------------------------------------------------------------------------------
\begin{homeworkProblem}
  
    We found that the 3rd value of $\bm{\alpha}$ is zero. The support vectors
    are the 1st, 2nd, and 4th values of $\alpha$. We used the dual solution in
    Problem 1, see the same code at the end of the homework.

\end{homeworkProblem}
%===============================================================================

%===============================================================================
%-------------------------------------------------------------------------------
%	PROBLEM 3
%-------------------------------------------------------------------------------
\begin{homeworkProblem}

    We find that changing $t_0 = 0$ changes the classification $\bm{\hat{x}} =
    0$, but we still classify the people 100\% correctly. This likely means
    that the classification is robust to the particular hinge loss chosen.

\end{homeworkProblem}
%===============================================================================

%===============================================================================
%-------------------------------------------------------------------------------
%	PROBLEM 4
%-------------------------------------------------------------------------------
\begin{homeworkProblem}

    We performed a simulation of generating datasets of sizes 10$\times$10,
    100$\times$100, 1000$\times$1000. For each dataset size, we ran 100
    iterations of regenerating the data, and classified the generated data with
    the least squares, Gaussian kernel SVMs, and polynomial kernel SVMs. We
    calculated the percent total of mistakes made by each of the three methods,
    see Table~\ref{table:prob4} for the results. See Figure~\ref{fig:prob4} for
    an example of classifications for a 1000$\times$1000 dataset. See the code
    at the end of the homework.

    \begin{table}[!ht]

        \caption{\label{table:prob4} Resulting \% total classification mistakes
        for different methods and dataset sizes.}

        \begin{center}
            \begin{tabular}{lccc}
                
                & 10$\times$10 & 100$\times$100 & 1000$\times$1000 \\
                \hline \hline
                Least Squares     & 33 & 45 & 48.4 \\
                Gaussian Kernel   & 2  & 2 & 1.3 \\
                Polynomial Kernel & 8  & 3 & 3.9 \\

            \end{tabular}
        \end{center}
    \end{table}
    \begin{figure}[!ht]
        
        \begin{centering}
        
        \includegraphics[width=\linewidth]{fig_prob4.png}

        \caption{\label{fig:prob4} Results from classifications of the training
    data. The Gaussian kernel outperforms the least squares and polynomial
    kernel classification.} \end{centering}

    \end{figure} 

\end{homeworkProblem}
%===============================================================================

\clearpage
{\huge Code:}

{\large \bf Problem 1 \& 2} \\
\lstinputlisting{hw8_prob1.m} 
\hrule \hrule

{\large \bf Problem 3} \\
\lstinputlisting{hw8_prob3.m} 
\hrule \hrule

{\large \bf Problem 4} \\
\lstinputlisting{hw8_prob4.m} 
\hrule \hrule

\end{document}

