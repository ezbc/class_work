%-------------------------------------------------------------------------------
%	PACKAGES AND OTHER DOCUMENT CONFIGURATIONS
%-------------------------------------------------------------------------------

\documentclass{article}

% Packages
% packages to load
\usepackage{amssymb} %
\usepackage[utf8]{inputenc}
%\usepackage[margin=1in]{geometry}
\usepackage{graphics,graphicx} % jpg and pdf figures
%\usepackage{sidecap}
\usepackage{amsmath}
\usepackage{verbatim} % block comments
\usepackage{multicol, caption} % multicolumn formating
\usepackage{lipsum} % needed for figures in multicol format
\usepackage{float} % for figure placement
\usepackage{natbib} % for bibtex referencing
\usepackage{cite} % cite with bibtex
\usepackage{aas_macros} % for journal referencing





% formatting
%%%%%%%%%%%%%%%%%%%%%%%%%%%%
%%%%%% Page dimensions %%%%%
%%%%%%  DO NOT CHANGE  %%%%%
%%%%%%%%%%%%%%%%%%%%%%%%%%%%

\textheight=247mm
\textwidth=180mm
\topmargin=-7mm
\oddsidemargin=-10mm
\evensidemargin=-10mm
\parindent 10pt

% Citation format
\bibliographystyle{apj}
\citestyle{aa}



%----------------------------------------------------------------------------------------
%	NAME AND CLASS SECTION
%----------------------------------------------------------------------------------------

\newcommand{\hmwkTitle}{Homework 2} % Assignment title
\newcommand{\hmwkDueDate}{Friday, Sept. 25} % Due date
\newcommand{\hmwkClass}{ECE 532} % Course/class
\newcommand{\hmwkClassTime}{11:00 am} % Class/lecture time
\newcommand{\hmwkClassInstructor}{Robert Nowak} % Teacher/lecturer
\newcommand{\hmwkAuthorName}{Elijah Bernstein-Cooper} % Your name

%-------------------------------------------------------------------------------
%	TITLE PAGE
%-------------------------------------------------------------------------------

\title{\vspace{0in}
    \textmd{\textbf{\hmwkClass:\ \hmwkTitle}}\\
    \normalsize\vspace{0.1in}\small{Due\ on\ \hmwkDueDate}\\
    \vspace{0.1in}\large{\textit{\hmwkClassInstructor\ \hmwkClassTime}}
    \vspace{0.5in}}

\author{\textbf{Elijah Bernstein-Cooper}}
\date{\today} % Insert date here if you want it to appear below your name

%-------------------------------------------------------------------------------

\begin{document}

\maketitle
%\newpage

%===============================================================================
%-------------------------------------------------------------------------------
%	PROBLEM 1
%-------------------------------------------------------------------------------
\begin{homeworkProblem}

    \begin{homeworkSection}{1a}
        
        We know that for an overdetermined system, the least squares solution
        to the matrix $\bm{\hat{x}}$ will be 

        \begin{equation*}
            {(\bm{A}^T \bm{A})}^{-1}\bm{A}^T\bm{b}
        \end{equation*}

        \noindent which gives us

        \begin{equation*}
            \bm{\hat{x}} = 
                0.75 \left(\begin{matrix} 1 \\ 1 \\ 1 \end{matrix}\right) + 
                0.25 \left(\begin{matrix} 1 \\ -1 \\ 1 \end{matrix}\right)
        \end{equation*}

    \end{homeworkSection}

    \begin{homeworkSection}{1b}
        
        See Figure~\ref{fig:1b} for a sketch of the problem.

        \begin{figure}[!ht]
            \begin{centering}

            \includegraphics[scale=0.08]{problem1b_fig.png}

            \caption{\label{fig:1b} Sketch of $\bm{A}$ and $\bm{b}$.}

            \end{centering}
        \end{figure}

    \end{homeworkSection}

\end{homeworkProblem}
%\clearpage
%===============================================================================

%===============================================================================
%-------------------------------------------------------------------------------
%	PROBLEM 2 
%-------------------------------------------------------------------------------
\begin{homeworkProblem}

    \begin{homeworkSection}{2a}
        
        The matrix A will be a $m \times 2$ matrix consisting of the measured
        responses for each condition in $m$ experiments.

    \end{homeworkSection}

    \begin{homeworkSection}{2c}

                See Figure~\ref{fig:2c}.

        \begin{figure}[!ht]
            \begin{centering}
            \includegraphics[scale=1]{problem2c_fig.png}

            \caption{\label{fig:2c} $\bm{A}$ and $\bm{b}$.}
            
            \end{centering}
        \end{figure}

    \end{homeworkSection}

    \begin{homeworkSection}{2d}

                See Figure~\ref{fig:2d}.

        \begin{centering}
        \begin{figure}[!ht]
            \includegraphics[scale=1]{problem2d_fig.png}

            \caption{\label{fig:2d} $\bm{A}$ and $\bm{b}$.}

        \end{figure}
        \end{centering}

    \end{homeworkSection}

\end{homeworkProblem}
%===============================================================================

%===============================================================================
%-------------------------------------------------------------------------------
%	PROBLEM 3
%-------------------------------------------------------------------------------
\begin{homeworkProblem}

    \begin{homeworkSection}{3a}

        To find min$_{\bm{x}} \|\tilde{\bm{A}}\bm{x} - \tilde{\bm{b}}\|_2$ we
        know that the expression will be smallest when the difference inside
        the norm is equal to 0. Thus our problem becomes solving the linear
        equation

        \begin{equation*}
            \tilde{\bm{A}}\bm{x} - \tilde{\bm{b}}
        \end{equation*}

        \noindent for $\bm{x}$. 

    \end{homeworkSection}

\end{homeworkProblem}
%===============================================================================

%===============================================================================
%-------------------------------------------------------------------------------
%	PROBLEM 4
%-------------------------------------------------------------------------------
\begin{homeworkProblem}

    \begin{homeworkSection}{4a}

        By solving the linear equation

        \begin{equation*}
            \bm{A x} = \bm{b}
        \end{equation*}

        \noindent we obtain a set of weights $\bm{x}$ where

        \begin{equation*}
            \bm{x} = \left(\begin{matrix}
                             0.943 \\
                             0.213 \\
                             0.266 \\
                            -0.392 \\
                            -0.005 \\
                            -0.017 \\
                            -0.166 \\
                            -0.082 \\
                            -0.166 \\
          \end{matrix}\right)       
        \end{equation*}

    \end{homeworkSection}

    \begin{homeworkSection}{4b}

        To classify a new face as happy or mad we would multiply the new data
        matrix $\bm{a}_i$ by the weights. If the product is greater than 0,
        then the face is happy, if the product is less than 0, the face is
        mad.

    \end{homeworkSection}

    \begin{homeworkSection}{4c}

        Feature 1 best determines a happy face because it has the largest
        positive weight, while feature 4 best determines a mad face because it
        has the smallest negative weight. 

    \end{homeworkSection}

    \begin{homeworkSection}{4d}
        
        To build a classifier out of just three features we will use the
        features with the largest absolute value weight, at least one feature
        for happy and mad (positive and negative weights) and attempt to
        minimize the difference between the sum of the absolute values of
        weights for happy and mad features. The final criterion allows us to
        compromise our accuracy for determining a happy face or a mad face. \\

        These criteria lead us to choose features 1, 4, and 9 as the three most
        important features for determining a happy or a mad face. We would
        build a classifier in the same way as in Problem \S 4b, except using
        only features 1, 4, and 9.

    \end{homeworkSection}
    
    \begin{homeworkSection}{4e}

    \end{homeworkSection}
    
    \begin{homeworkSection}{4f}

    \end{homeworkSection}

\end{homeworkProblem}
%===============================================================================

\end{document}

