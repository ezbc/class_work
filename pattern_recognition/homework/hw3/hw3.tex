%-------------------------------------------------------------------------------
%	PACKAGES AND OTHER DOCUMENT CONFIGURATIONS
%-------------------------------------------------------------------------------

\documentclass{article}

% Packages
% packages to load
\usepackage{amssymb} %
\usepackage[utf8]{inputenc}
%\usepackage[margin=1in]{geometry}
\usepackage{graphics,graphicx} % jpg and pdf figures
%\usepackage{sidecap}
\usepackage{amsmath}
\usepackage{verbatim} % block comments
\usepackage{multicol, caption} % multicolumn formating
\usepackage{lipsum} % needed for figures in multicol format
\usepackage{float} % for figure placement
\usepackage{natbib} % for bibtex referencing
\usepackage{cite} % cite with bibtex
\usepackage{aas_macros} % for journal referencing





% formatting
%%%%%%%%%%%%%%%%%%%%%%%%%%%%
%%%%%% Page dimensions %%%%%
%%%%%%  DO NOT CHANGE  %%%%%
%%%%%%%%%%%%%%%%%%%%%%%%%%%%

\textheight=247mm
\textwidth=180mm
\topmargin=-7mm
\oddsidemargin=-10mm
\evensidemargin=-10mm
\parindent 10pt

% Citation format
\bibliographystyle{apj}
\citestyle{aa}



%----------------------------------------------------------------------------------------
%	NAME AND CLASS SECTION
%----------------------------------------------------------------------------------------

\newcommand{\hmwkTitle}{Homework 2} % Assignment title
\newcommand{\hmwkDueDate}{Friday, Sept. 25} % Due date
\newcommand{\hmwkClass}{ECE 532} % Course/class
\newcommand{\hmwkClassTime}{11:00 am} % Class/lecture time
\newcommand{\hmwkClassInstructor}{Robert Nowak} % Teacher/lecturer
\newcommand{\hmwkAuthorName}{Elijah Bernstein-Cooper} % Your name

%-------------------------------------------------------------------------------
%	TITLE PAGE
%-------------------------------------------------------------------------------

\title{\vspace{0in}
    \textmd{\textbf{\hmwkClass:\ \hmwkTitle}}\\
    \normalsize\vspace{0.1in}\small{Due\ on\ \hmwkDueDate}\\
    \vspace{0.1in}\large{\textit{\hmwkClassInstructor\ \hmwkClassTime}}
    \vspace{0.5in}}

\author{\textbf{Elijah Bernstein-Cooper}}
\date{\today} % Insert date here if you want it to appear below your name

%-------------------------------------------------------------------------------

\begin{document}

\maketitle
%\newpage

%===============================================================================
%-------------------------------------------------------------------------------
%	PROBLEM 1
%-------------------------------------------------------------------------------
\begin{homeworkProblem}

    \begin{homeworkSection}{1a}
        
        We know that for an overdetermined system, the least squares solution
        to the matrix $\bm{\hat{x}}$ will be 

        \begin{equation*}
            {(\bm{A}^T \bm{A})}^{-1}\bm{A}^T\bm{b}
        \end{equation*}

        \noindent which gives us

        \begin{equation*}
            \bm{\hat{x}} = 
                0.75 \left(\begin{matrix} 1 \\ 1 \\ 1 \end{matrix}\right) + 
                0.25 \left(\begin{matrix} 1 \\ -1 \\ 1 \end{matrix}\right)
        \end{equation*}

    \end{homeworkSection}

    \begin{homeworkSection}{1b}
        
        See Figure~\ref{fig:1b} for a sketch of the problem.

        \begin{centering}
        \begin{figure}[!ht]
            \includegraphics[scale=0.08]{problem1b_fig.jpg}

            \caption{\label{fig:1b} Sketch of $\bm{A}$ and $\bm{b}$.}

        \end{figure}
        \end{centering}

    \end{homeworkSection}

\end{homeworkProblem}
%\clearpage
%===============================================================================

%===============================================================================
%-------------------------------------------------------------------------------
%	PROBLEM 2 
%-------------------------------------------------------------------------------
\begin{homeworkProblem}

    \begin{homeworkSection}{2a}
        
        The matrix A will be a $m \times 2$ matrix consisting of the measured
        responses for each condition in $m$ experiments.

    \end{homeworkSection}

    \begin{homeworkSection}{2b}

        

    \end{homeworkSection}

\end{homeworkProblem}
%===============================================================================

%===============================================================================
%-------------------------------------------------------------------------------
%	PROBLEM 3
%-------------------------------------------------------------------------------
\begin{homeworkProblem}

    Given 
    
    \begin{equation}
        \cos(\bm{x},\bm{y}) = \frac{\bm{x}^T\bm{y}}{\|\bm{x}\|_2 \|\bm{y}\|_2}
    \end{equation}

    \noindent and that $|\cos(\bm{x},\bm{y})| \leq 1$, the absolute value of
    the numerator cannot be larger than the denominator, thus $|\bm{x}^T\bm{y}|
    \leq \|\bm{x}\|_2 \|\bm{y}\|_2$.

\end{homeworkProblem}
%===============================================================================

%===============================================================================
%-------------------------------------------------------------------------------
%	PROBLEM 4
%-------------------------------------------------------------------------------
\begin{homeworkProblem}

    \begin{homeworkSection}{4a}

        Given $\bm{y} = \bm{Ax}$ we can write $\bm{x}$ as
        
        \begin{equation}
            \bm{x = A}^{-1}\bm{y}
        \end{equation}

    \end{homeworkSection}

    \begin{homeworkSection}{4b}

        To bound the 2-norm of $\bm{x}$ with a function of $\bm{A}$ and $\bm{y}$
        we first take $\|\bm{x}\| = \|\bm{A}^{-1}\bm{y}\|$ which can be
        expressed as

        \begin{equation}
            \frac{\|\bm{A}^{-1}\bm{y}\|}{\|\bm{y}\|}\|\bm{y}\|
        \end{equation}
        
        \noindent where

        \begin{equation}\label{eq:matrix_norm}
            \frac{\|\bm{A}^{-1}\bm{y}\|}{\|\bm{y}\|}
        \end{equation}

        \noindent is the matrix norm. Eq.~\ref{eq:matrix_norm} will always be
        less than $\|\bm{A}^{-1}\|$, thus
        
        \begin{equation}
            \|\bm{x}\| \leq \|\bm{A}^{-1}\|\|\bm{y}\|
        \end{equation}


    \end{homeworkSection}

\end{homeworkProblem}
%===============================================================================

%===============================================================================
%-------------------------------------------------------------------------------
%	PROBLEM 5
%-------------------------------------------------------------------------------
\begin{homeworkProblem}

    \begin{homeworkSection}{5a}

        The rank of $\bm{A}$ is 3.

    \end{homeworkSection}

    \begin{homeworkSection}{5b}

        $\bm{x}$ can be expressed as 

        \begin{equation}
            \bm{x} = \left(\begin{matrix}
                         0 &   0 &  1 \\
                         0 &   1 & -1 \\
                         1 &  -1 &  0
            \end{matrix} \right) \bm{y}
        \end{equation}

    \end{homeworkSection}

\end{homeworkProblem}
%===============================================================================

%===============================================================================
%-------------------------------------------------------------------------------
%	PROBLEM 6
%-------------------------------------------------------------------------------
\begin{homeworkProblem}

    \begin{homeworkSection}{6a}

        The rank of $\bm{X}$ is 3. 

    \end{homeworkSection}
        

    \begin{homeworkSection}{6b}
        
        The rank of $\frac{\bm{XX^T}}{n}$ is 3. 

    \end{homeworkSection}
    
    \begin{homeworkSection}{6c}

        A set of linearly independent columns of $\bm{X}$ are

        \begin{math}
             \left(\begin{matrix} 1\\0\\0\\0\\ \end{matrix}\right) 
             \left(\begin{matrix} 0\\1\\0\\0\\ \end{matrix}\right) 
             \left(\begin{matrix} 1\\0\\0\\1\\ \end{matrix}\right) 
        \end{math}

    \end{homeworkSection}

\end{homeworkProblem}
%===============================================================================



\end{document}

