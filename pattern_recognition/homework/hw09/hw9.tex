%-------------------------------------------------------------------------------
%	PACKAGES AND OTHER DOCUMENT CONFIGURATIONS
%-------------------------------------------------------------------------------

\documentclass{article}

% Packages
% packages to load
\usepackage{amssymb} %
\usepackage[utf8]{inputenc}
%\usepackage[margin=1in]{geometry}
\usepackage{graphics,graphicx} % jpg and pdf figures
%\usepackage{sidecap}
\usepackage{amsmath}
\usepackage{verbatim} % block comments
\usepackage{multicol, caption} % multicolumn formating
\usepackage{lipsum} % needed for figures in multicol format
\usepackage{float} % for figure placement
\usepackage{natbib} % for bibtex referencing
\usepackage{cite} % cite with bibtex
\usepackage{aas_macros} % for journal referencing





% formatting
%%%%%%%%%%%%%%%%%%%%%%%%%%%%
%%%%%% Page dimensions %%%%%
%%%%%%  DO NOT CHANGE  %%%%%
%%%%%%%%%%%%%%%%%%%%%%%%%%%%

\textheight=247mm
\textwidth=180mm
\topmargin=-7mm
\oddsidemargin=-10mm
\evensidemargin=-10mm
\parindent 10pt

% Citation format
\bibliographystyle{apj}
\citestyle{aa}



%-------------------------------------------------------------------------------
%	NAME AND CLASS SECTION
%-------------------------------------------------------------------------------

\newcommand{\hmwkTitle}{Homework 9} % Assignment title
\newcommand{\hmwkDueDate}{Wednesday, Nov 26} % Due date
\newcommand{\hmwkClass}{ECE 532} % Course/class
\newcommand{\hmwkClassTime}{11:00 am} % Class/lecture time
\newcommand{\hmwkClassInstructor}{Robert Nowak} % Teacher/lecturer
\newcommand{\hmwkAuthorName}{Elijah Bernstein-Cooper} % Your name

%-------------------------------------------------------------------------------
%	TITLE PAGE
%-------------------------------------------------------------------------------

\title{\vspace{0in}
    \textmd{\textbf{\hmwkClass:\ \hmwkTitle}}\\
    \normalsize\vspace{0.1in}\small{Due\ on\ \hmwkDueDate}\\
    \vspace{0.1in}\large{\textit{\hmwkClassInstructor\ \hmwkClassTime}}
    \vspace{0.5in}}

\author{\textbf{Elijah Bernstein-Cooper}}
\date{\today} % Insert date here if you want it to appear below your name

%-------------------------------------------------------------------------------

\begin{document}

\maketitle
%\newpage

%===============================================================================
%-------------------------------------------------------------------------------
%	PROBLEM 1
%-------------------------------------------------------------------------------
\begin{homeworkProblem}
  
    We clustered the data in right panel of Figure~\ref{fig:spectral}, using
    first kmeans++ and kmeans++ spectral clustering. We chose the number of
    clusters to be 2. In Figure~\ref{fig:kmeans} we show the classification of
    the data using kmeans++ clustering. We can see that kmeans++ prohibits a
    correct classification of the structured data, which we expect to be split
    as two concentric circles. Figure~\ref{fig:spectral} demonstrates that
    spectral kmeans++ provides the ability to correctly classify the groups in
    the data. This example result occurs every 1 out of 10 random samples. The
    spectral kmeans++ clustering is likely to fail on this dataset in correctly
    classifying the data, but has the potential to correctly classify the
    observations.

    We believe that this inaccuracy of the spectral kmeans++ clustering is due
    to the distance measurement we assign between points. Currently our
    distance is determined with a Gaussian distance

    \begin{equation}
        W_{ij} = {\rm exp}\left[-\frac{\|A_{i.} - A_{j.}\|^2_2}{2}\right]
    \end{equation}

    \noindent where $\bm{W}_{ij}$ is the $n \times n$ weight matrix, and
    $\bm{A}$ contains rows of coordinate pairs of the data. A distance
    measurement which more closely represented a point's distance from the
    center of the data would provide a more accurate spectral kmeans++
    classification. See the code used to complete this homework at the end.

    \begin{figure}[!ht]
        
        \begin{centering}
        
        \includegraphics[width=\linewidth]{hw9_fig_spectral_kmeans.png}

        \caption{\label{fig:spectral} Left: Original weight matrix. Middle:
        Spectral kmeans Laplacian matrix of data. Right: Data labeled by
    categorization of kmeans++ spectral clustering algorithm.} 
    
        \end{centering}

    \end{figure} 
   
    \begin{figure}[!ht]
        
        \begin{centering}
        
        \includegraphics[width=\linewidth]{hw9_fig_regular_kmeans.png}

        \caption{\label{fig:kmeans} Data labeled by categorization of kmeans++
        clustering algorithm. } 
    
        \end{centering}

    \end{figure} 


\end{homeworkProblem}
\clearpage
%===============================================================================

\clearpage
{\huge Code:}

{\large \bf Problem 1} \\
\lstinputlisting{hw9.m} 

\end{document}

