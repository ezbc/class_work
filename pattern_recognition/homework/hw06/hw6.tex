%-------------------------------------------------------------------------------
%	PACKAGES AND OTHER DOCUMENT CONFIGURATIONS
%-------------------------------------------------------------------------------

\documentclass{article}

% Packages
% packages to load
\usepackage{amssymb} %
\usepackage[utf8]{inputenc}
%\usepackage[margin=1in]{geometry}
\usepackage{graphics,graphicx} % jpg and pdf figures
%\usepackage{sidecap}
\usepackage{amsmath}
\usepackage{verbatim} % block comments
\usepackage{multicol, caption} % multicolumn formating
\usepackage{lipsum} % needed for figures in multicol format
\usepackage{float} % for figure placement
\usepackage{natbib} % for bibtex referencing
\usepackage{cite} % cite with bibtex
\usepackage{aas_macros} % for journal referencing





% formatting
%%%%%%%%%%%%%%%%%%%%%%%%%%%%
%%%%%% Page dimensions %%%%%
%%%%%%  DO NOT CHANGE  %%%%%
%%%%%%%%%%%%%%%%%%%%%%%%%%%%

\textheight=247mm
\textwidth=180mm
\topmargin=-7mm
\oddsidemargin=-10mm
\evensidemargin=-10mm
\parindent 10pt

% Citation format
\bibliographystyle{apj}
\citestyle{aa}



%-------------------------------------------------------------------------------
%	NAME AND CLASS SECTION
%-------------------------------------------------------------------------------

\newcommand{\hmwkTitle}{Homework 6} % Assignment title
\newcommand{\hmwkDueDate}{Saturday, Nov 1} % Due date
\newcommand{\hmwkClass}{ECE 532} % Course/class
\newcommand{\hmwkClassTime}{11:00 am} % Class/lecture time
\newcommand{\hmwkClassInstructor}{Robert Nowak} % Teacher/lecturer
\newcommand{\hmwkAuthorName}{Elijah Bernstein-Cooper} % Your name

%-------------------------------------------------------------------------------
%	TITLE PAGE
%-------------------------------------------------------------------------------

\title{\vspace{0in}
    \textmd{\textbf{\hmwkClass:\ \hmwkTitle}}\\
    \normalsize\vspace{0.1in}\small{Due\ on\ \hmwkDueDate}\\
    \vspace{0.1in}\large{\textit{\hmwkClassInstructor\ \hmwkClassTime}}
    \vspace{0.5in}}

\author{\textbf{Elijah Bernstein-Cooper}}
\date{\today} % Insert date here if you want it to appear below your name

%-------------------------------------------------------------------------------

\begin{document}

\maketitle
%\newpage

%===============================================================================
%-------------------------------------------------------------------------------
%	PROBLEM 1
%-------------------------------------------------------------------------------
\begin{homeworkProblem}

    \begin{homeworkSection}{1a}

        We partitioned the face emotion data into 8 random partitions, then for
        each permutation of using 6 partitions for training, 1 partition for
        tuning and 1 partition for testing, we computed the truncated SVD. To
        compute the truncated SVD we first took the SVD of the training data,
        then adjusted the regularization parameter, $k$, until the residual,
        $\|\bm{b} - \bm{Ax}\|^2$, was minimized. The median $k$ was 7. The
        truncated SVD was then used to compute an estimate of the error in
        prediction with the test partition. We found that for our 56
        combinations of partitions, the average predictive error in from the
        SVD was 1.9. See the code at end of homework used to complete this
        problem.

    \end{homeworkSection}

    \begin{homeworkSection}{1b}

        We partitioned the face emotion data and computed a regularized least
        squares solutions in a similar manner to \S\,1a. To compute the
        regularized least squares, we maximize $\|\bm{b} - \bm{Ax}\|^2 +
        \lambda\|x\|^2$, with the solution $\bm{\hat{x}} = (A^T A + \lambda
        I)^{-1} A^T b$. We cycle through logarithmically spaced $\lambda$ until
        our maximization is completed. The median $\lambda$ was 2. The RLS was
        then used to compute an estimate of the error in prediction with the
        test partition. We found that for our 56 combinations of partitions,
        the average predictive error in from the RLS was 1.9, similar to the
        truncated SVD predictive error.  See the code at end of homework used
        to complete this problem.

    \end{homeworkSection}

\end{homeworkProblem}
\clearpage
%===============================================================================

%===============================================================================
%-------------------------------------------------------------------------------
%	PROBLEM 2 
%-------------------------------------------------------------------------------
\begin{homeworkProblem}
    
    \begin{homeworkSection}{2a}    
    
        We attempt to derive an original signal from noisy, blurred data. We
        used the cross-validation technique to compute the regularization
        parameters of the truncated SVD, and regularized least squares (RLS)
        solutions. We also compute a simple least-squares solution of the
        signal. See Figure~\ref{fig:prob2a} for the best-estimate signals from
        each method. See the code at end of homework used to complete this
        problem.

        \begin{figure}
            
            \includegraphics[width=\linewidth]{prob2a_fig.png}

            \caption{\label{fig:prob2a} The original signal plotted with the
            estimates from the least-squares, regularized least-squares, and
        truncated SVD methods of the signal.}

        \end{figure} 
    \end{homeworkSection}
    
    \begin{homeworkSection}{2b}

        We computed the 2-norms of the LS, RLS, and truncated SVD solutions for
        $\bm{\hat{x}}$ for varying values of noise strengths and averaging
        widths. See Table~\ref{table:1} for results. In general we find that
        the truncated SVD performs better than the regularized least squares
        method when comparing the estimated signal with the true signal. See
        the code at end of homework used to complete this problem.

        \begin{table}[H]

            \caption{\label{table:1} Fitting results for different averaging
            size kernels, $k$, and data noise strengths, $\sigma$. The norms of
        the LS, RLS and SVDs are all 2-norms between the blurred signal and the
    best-estimate of the blurred signal.}

            \begin{center}
                \begin{tabular}{lccc}

                    Properties & LS norm & RLS norm & SVD norm \\
                    \hline \hline
                    k = 30, $\sigma$ = 0.1 & 80 & 12.0 & 8.1 \\
                    k = 10, $\sigma$ = 0.1 & 41 & 9.5 & 7.9 \\
                    k = 50, $\sigma$ = 0.1 & 120 & 13.3 & 13.0 \\
                    k = 30, $\sigma$ = 0.01 & 9 & 6.3 & 6.7 \\
                    k = 30, $\sigma$ = 1 & 955 & 15.2 & 10.8 \\

                \end{tabular}
            \end{center}
        \end{table}

    \end{homeworkSection}
    
\end{homeworkProblem}
\clearpage
%===============================================================================

Code:

{\large \bf Problem 1a} \\
\lstinputlisting{hw6_prob1a.m} 
\hrule \hrule

{\large \bf Problem 1b} \\
\lstinputlisting{hw6_prob1b.m} 
\hrule \hrule

{\large \bf Problem 2a+b} \\
\lstinputlisting{hw6_prob2a.m}

\end{document}

