%-------------------------------------------------------------------------------
%	PACKAGES AND OTHER DOCUMENT CONFIGURATIONS
%-------------------------------------------------------------------------------

\documentclass{article}

% Packages
% packages to load
\usepackage{amssymb} %
\usepackage[utf8]{inputenc}
%\usepackage[margin=1in]{geometry}
\usepackage{graphics,graphicx} % jpg and pdf figures
%\usepackage{sidecap}
\usepackage{amsmath}
\usepackage{verbatim} % block comments
\usepackage{multicol, caption} % multicolumn formating
\usepackage{lipsum} % needed for figures in multicol format
\usepackage{float} % for figure placement
\usepackage{natbib} % for bibtex referencing
\usepackage{cite} % cite with bibtex
\usepackage{aas_macros} % for journal referencing





% formatting
%%%%%%%%%%%%%%%%%%%%%%%%%%%%
%%%%%% Page dimensions %%%%%
%%%%%%  DO NOT CHANGE  %%%%%
%%%%%%%%%%%%%%%%%%%%%%%%%%%%

\textheight=247mm
\textwidth=180mm
\topmargin=-7mm
\oddsidemargin=-10mm
\evensidemargin=-10mm
\parindent 10pt

% Citation format
\bibliographystyle{apj}
\citestyle{aa}



%----------------------------------------------------------------------------------------
%	NAME AND CLASS SECTION
%----------------------------------------------------------------------------------------

\newcommand{\hmwkTitle}{Homework 1} % Assignment title
\newcommand{\hmwkDueDate}{Tuesday, Sept. 9} % Due date
\newcommand{\hmwkClass}{ECE 532} % Course/class
\newcommand{\hmwkClassTime}{11:00 am} % Class/lecture time
\newcommand{\hmwkClassInstructor}{Robert Nowak} % Teacher/lecturer
\newcommand{\hmwkAuthorName}{Elijah Bernstein-Cooper} % Your name

%-------------------------------------------------------------------------------
%	TITLE PAGE
%-------------------------------------------------------------------------------

\title{\vspace{0in}
    \textmd{\textbf{\hmwkClass:\ \hmwkTitle}}\\
    \normalsize\vspace{0.1in}\small{Due\ on\ \hmwkDueDate}\\
    \vspace{0.1in}\large{\textit{\hmwkClassInstructor\ \hmwkClassTime}}
    \vspace{0.5in}}

\author{\textbf{Elijah Bernstein-Cooper}}
\date{\today} % Insert date here if you want it to appear below your name

%-------------------------------------------------------------------------------

\begin{document}

\maketitle
%\newpage

%===============================================================================
%-------------------------------------------------------------------------------
%	PROBLEM 1
%-------------------------------------------------------------------------------
\begin{homeworkProblem}

    \textbf{Design an algorithm to find the genes that are most predictive of a
        disease or phenotype. DNA microarrays can be used to measure the amount
        of protein produced by each gene (the so-called "gene expression
        level").  Suppose we measure the gene expression levels of a number of
    people with and without a particular disease.  Assume m genes are
measured.  What is your data matrix A?  How would you decide which genes are
most important to or involved in the disease process?} \\
    
    Our data matrix, $A$, will be a $k \times m$ matrix, with $k$ people and
    $m$ genes. To determine which genes are most likely to cause disease we
    first assign a label matrix $L$, a column vector of length $k$ where the
    $i^{\rm th}$ element corresponds to the label for the $i^{\rm th}$ person
    in our matrix $A$. Diseased people will be labeled 1, healthy people will
    be labeled 0. \\

    Our problem will then be reduced to solving the linear equation

    \begin{equation}
        A x = L
    \end{equation}

    \noindent where $x$ will be a column vector of length $m$. We then solve
    for $x$. $x$ will describe the relative contributions of genes to
    disease or health.

\end{homeworkProblem}
%\clearpage
%===============================================================================

%===============================================================================
%-------------------------------------------------------------------------------
%	PROBLEM 2 
%-------------------------------------------------------------------------------
\begin{homeworkProblem}
    
    \textbf{Now suppose we have gene expression data for n different strains of
        yeast.  We also measure phenotypic similarities between the different
        strains (e.g., similarity measures based on the observable
        characteristics or traits of the strains like shaper or color).  In
        other words, we have an n x n matrix of similarity values, say ranging
        continuously between 0 (dissimilar) to 1 (identical).  How would you
        determine which genes are most important for predicting phenotypic
    similarities?} \\

    We have an $n \times n$ matrix $S$ with similarity values between the
    $i^{\rm th}$ yeast strain and the $j^{\rm th}$ yeast strain. The matrix
    will be symmetric about the diagonal. We can compute a singular value
    decomposition of $S$, whereby we can rank the singular values to rank the
    phenotypical similarities.
    
    
    \begin{comment}
    We also have an $n \times m$ matrix
    $G$ corresponding to the genotype data for each yeast strain. To determine
    which genes are most important for predicting phenotype similarities, we
    have the following

    \begin{equation}
        G x = S
    \end{equation}

    \noindent where $x$ will be an $m \times n$ matrix where each $i^{\rm th}$
    value corresponds to the gene and the $j^{\rm th}$ yeast strain with the
    given phenotypes.

    \end{comment}



    %\begin{homeworkSection}{i)}
   

    %\end{homeworkSection}

\end{homeworkProblem}

\end{document}

