%-------------------------------------------------------------------------------
%	PACKAGES AND OTHER DOCUMENT CONFIGURATIONS
%-------------------------------------------------------------------------------

\documentclass{article}

% Packages
% packages to load
\usepackage{amssymb} %
\usepackage[utf8]{inputenc}
%\usepackage[margin=1in]{geometry}
\usepackage{graphics,graphicx} % jpg and pdf figures
%\usepackage{sidecap}
\usepackage{amsmath}
\usepackage{verbatim} % block comments
\usepackage{multicol, caption} % multicolumn formating
\usepackage{lipsum} % needed for figures in multicol format
\usepackage{float} % for figure placement
\usepackage{natbib} % for bibtex referencing
\usepackage{cite} % cite with bibtex
\usepackage{aas_macros} % for journal referencing





% formatting
%%%%%%%%%%%%%%%%%%%%%%%%%%%%
%%%%%% Page dimensions %%%%%
%%%%%%  DO NOT CHANGE  %%%%%
%%%%%%%%%%%%%%%%%%%%%%%%%%%%

\textheight=247mm
\textwidth=180mm
\topmargin=-7mm
\oddsidemargin=-10mm
\evensidemargin=-10mm
\parindent 10pt

% Citation format
\bibliographystyle{apj}
\citestyle{aa}



%----------------------------------------------------------------------------------------
%	NAME AND CLASS SECTION
%----------------------------------------------------------------------------------------

\newcommand{\hmwkTitle}{Homework 3} % Assignment title
\newcommand{\hmwkDueDate}{Thursday, Oct 2} % Due date
\newcommand{\hmwkClass}{ECE 532} % Course/class
\newcommand{\hmwkClassTime}{11:00 am} % Class/lecture time
\newcommand{\hmwkClassInstructor}{Robert Nowak} % Teacher/lecturer
\newcommand{\hmwkAuthorName}{Elijah Bernstein-Cooper} % Your name

%-------------------------------------------------------------------------------
%	TITLE PAGE
%-------------------------------------------------------------------------------

\title{\vspace{0in}
    \textmd{\textbf{\hmwkClass:\ \hmwkTitle}}\\
    \normalsize\vspace{0.1in}\small{Due\ on\ \hmwkDueDate}\\
    \vspace{0.1in}\large{\textit{\hmwkClassInstructor\ \hmwkClassTime}}
    \vspace{0.5in}}

\author{\textbf{Elijah Bernstein-Cooper}}
\date{\today} % Insert date here if you want it to appear below your name

%-------------------------------------------------------------------------------

\begin{document}

\maketitle
%\newpage

%===============================================================================
%-------------------------------------------------------------------------------
%	PROBLEM 1
%-------------------------------------------------------------------------------
\begin{homeworkProblem}

    \begin{homeworkSection}{1a}

        Our basis vectors, $\bm{x_1}$ and $\bm{x_2}$ will be

        \begin{equation*}
            \bm{q_1} = \bm{a}_1 \qquad \bm{q_2} =
            \frac{\bm{a}_2^\prime}{\|\bm{a}_2^\prime\|}, \qquad
            \bm{a}_2^\prime = \bm{a}_2 - (\bm{q}_1^T \bm{a}_2)\bm{q}_1
        \end{equation*}

        \noindent which gives us

        \begin{equation*}
            \bm{q}_1 = \begin{matrix} 1\\0\\0\\ \end{matrix}
        \end{equation*}

        \begin{equation*}
        \bm{q}_2 = \frac{ \left(\begin{matrix} 1\\3\\4\\ \end{matrix}\right) -
            \left(\left(\begin{matrix} 1&0&0 \end{matrix}\right)
                       \left(\begin{matrix} 1\\3\\4\\ \end{matrix}\right)\right)
               \left(\begin{matrix} 1\\0\\0\\\end{matrix}\right)}
                   {\|\bm{a}_2^\prime\|}
        \end{equation*}

        \begin{equation*}
            \bm{q}_2 = \left(\begin{matrix} 0\\0.6\\0.8\\\end{matrix}\right)
        \end{equation*}

    \end{homeworkSection}

    \begin{homeworkSection}{1b}

        See Figure~\ref{fig:1}

        \begin{figure}[!ht]
        \begin{centering}
            \includegraphics[scale=0.07]{problem1b_fig.png}

            \caption{\label{fig:1} Sketch of spans of $\bm{Q}$ and $\bm{A}$.}

        \end{centering}
        \end{figure}

    \end{homeworkSection}

    \begin{homeworkSection}{1c}

        \begin{equation*}
        \hat{\bm{b}} = \left(\begin{matrix} 1\\1.5\\2.1\\\end{matrix}\right)
        \end{equation*}

        \pythonexternal{hw4_prob1.m}

    \end{homeworkSection}

\end{homeworkProblem}
%\clearpage
%===============================================================================

%===============================================================================
%-------------------------------------------------------------------------------
%	PROBLEM 2 
%-------------------------------------------------------------------------------
\begin{homeworkProblem}

    \pythonexternal{gs_ortho.m}

\end{homeworkProblem}
%===============================================================================

%===============================================================================
%-------------------------------------------------------------------------------
%	PROBLEM 3
%-------------------------------------------------------------------------------
\begin{homeworkProblem}

    The code below produces the same number of linearly independent basis
    vectors as the rank of each matrix.

    \pythonexternal{hw4_prob3.m}

\end{homeworkProblem}
%===============================================================================

%===============================================================================
%-------------------------------------------------------------------------------
%	PROBLEM 4
%-------------------------------------------------------------------------------
\begin{homeworkProblem}

    \begin{homeworkSection}{4a}
        
        $\hat{\bm{b}}$ can be estimated with the orthonormal vectors by

        \begin{equation*}
            \hat{\bm{b}} = \bm{Q}\bm{Q}^T\bm{b}
        \end{equation*}

        \noindent which compared with the least squares solution of
        $\hat{\bm{b}}$ with the Pearson's correlation coefficient we get a
        correlation of 1. 
    
    \end{homeworkSection}

    \begin{homeworkSection}{4b}

        The orthonormal vectors are the same for the Gram-Schmidt code and the
        orth algorithm. The pearson correlation coefficients between the
        resulting basis vectors is 1.

    \end{homeworkSection}

    \pythonexternal{hw4_prob4.m}
\end{homeworkProblem}
%===============================================================================

\end{document}

