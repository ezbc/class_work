%%%%%%%%%%%%%%%%%%%%%%%%%%%%%%%%%%%%%%%%%%%%%%%%%%%%%%%%%%%%%%
%% LaTeX template for the science justification to be       %%
%%       submitted as part of an ALMA proposal.             %%
%%                                                          %%
%%                      ALMA Cycle 2                        %%
%%                                                          %%
%%%%%%%%%%%%%%%%%%%%%%%%%%%%%%%%%%%%%%%%%%%%%%%%%%%%%%%%%%%%%%

%%%%%%%%%%%%%%%%%%%%%%%%%%%%%%%%%%%%%%%%%%%%%%%%%%
%%%%% How to convert this document to PDF %%%%%%%%
%%%%%%%%%%%%%%%%%%%%%%%%%%%%%%%%%%%%%%%%%%%%%%%%%%

% If your figures are stored as PostScript files, you can use the 
% following commands to generate a PDF file of your proposal:

%% latex file.tex
%% dvips file.dvi
%% ps2pdf file.ps file.pdf 


% If your figures are PDF images or bitmap pictures in PNG, JPG, or GIF format,
% you can use the pdflatex command to generate a PDF file from this template
% (note, however, that the pdflatex command does not handle PostScript files):

% pdflatex file.tex


% WARNINGS: 
%           1. You must make sure that PDF output generated from this
%              template is complete both when displayed with a viewer 
%              (acroread, for example) and when printed on paper.
%              LaTeX installations vary greatly and therefore it might 
%              not be possible to get all proposals to come out 
%              correctly with a single text page layout. 
%              In some cases you will have to adjust the 
%              \topmargin=-7mm command in the template to center the 
%              text vertically in the page.  
%           2. The scientific justification, figures, tables, references,
%              and public outreach statement must all fit within the
%              4-page limit.
%           3. You are free to include colour images in your proposal 
%              justification. Proposals are distributed to ALMA Review Panels 
%              in electronic form. However, the scientific content of the 
%              images should still remain clear when displayed or printed
%              in black and white.

%%%%%%%%%%%%%%%%%%%%%%%%%%%%%%%%%%%%%%%%%%%%%%
%%%%% Default format: 12pt single column %%%%%
%%%%%%%%%%%%%%%%%%%%%%%%%%%%%%%%%%%%%%%%%%%%%%

\documentclass[12pt,a4paper]{article}

\usepackage{graphics,graphicx}

% Load commands

\newcommand{\HRule}{\rule{\linewidth}{0.5mm}}
\newcommand{\Hrule}{\rule{\linewidth}{0.3mm}}
\newcommand{\ergseccm}{erg\,s$^{-1}$\,cm$^{-2}$}
\newcommand{\etal}{et\,al.}
\newcommand{\halpha}{H$\alpha$}
\newcommand{\lya}{Ly$\alpha$}
\newcommand{\lsim}{\raise0.3ex\hbox{$<$}\kern-0.75em{\lower0.65ex\hbox{$\sim$}}}
\newcommand{\mjpbeam}{\,\,mJy\,beam$^{-1}$}
\newcommand{\msun}{M$_{\odot}$}

% column densities
\newcommand{\hii}{H\,{\sc ii}}
\newcommand{\hi}{{\rm H\,{\sc i}}}
\newcommand{\nhi}{{$N$(\rm H\,{\sc i})}}
\newcommand{\hiColDens}{{$N($\rm H$_{\rm 2})$}}
\newcommand{\htwo}{{\rm H$_{\rm 2}$}}
\newcommand{\htwoColDens}{{$N($\rm H$_{\rm 2})$}}
\newcommand{\hisd}{$\Sigma_{\rm H\,{\text{\sc i}}}$}
\newcommand{\htwosd}{$\Sigma_{\rm H2}$}
\newcommand{\hsd}{$\Sigma_{\rm H}$}


\newcommand{\ii}{{\sc ii}}
\newcommand{\iii}{{\sc iii}}

% Units
\newcommand{\kms}{km\,s$^{-1}$}
\newcommand{\pom}{\,$\pm$\,}
\newcommand{\mm}{$\mu$m}
\newcommand{\lcdm}{$\Lambda$CDM}
\newcommand{\cm}{cm$^{-2}$}
\newcommand{\colDens}{$\times\ 10^{20}$ cm$^{-2}$}

% Units
\newcommand{\vrot}{$v_{\rm rot}$}
\newcommand{\vdisp}{$\sigma_{\rm HI}$}
\newcommand{\tspin}{$T_{\rm s}$}
\newcommand{\tdust}{$T_{\rm d}$}



% load packages
% packages to load
\usepackage{amssymb} %
\usepackage[utf8]{inputenc}
%\usepackage[margin=1in]{geometry}
\usepackage{graphics,graphicx} % jpg and pdf figures
%\usepackage{sidecap}
\usepackage{amsmath}
\usepackage{verbatim} % block comments
\usepackage{multicol, caption} % multicolumn formating
\usepackage{lipsum} % needed for figures in multicol format
\usepackage{float} % for figure placement
\usepackage{natbib} % for bibtex referencing
\usepackage{cite} % cite with bibtex
\usepackage{aas_macros} % for journal referencing





% load formatting
%%%%%%%%%%%%%%%%%%%%%%%%%%%%
%%%%%% Page dimensions %%%%%
%%%%%%  DO NOT CHANGE  %%%%%
%%%%%%%%%%%%%%%%%%%%%%%%%%%%

\textheight=247mm
\textwidth=180mm
\topmargin=-7mm
\oddsidemargin=-10mm
\evensidemargin=-10mm
\parindent 10pt

% Citation format
\bibliographystyle{apj}
\citestyle{aa}



%%%%%%%%%%%%%%%%%%%%%%%%%%%%%
%%%%% Start of document %%%%% 
%%%%%%%%%%%%%%%%%%%%%%%%%%%%%

\begin{document}
\pagestyle{plain}
\pagenumbering{arabic}
 
%%%%%%%%%%%%%%%%%%%%%%%%%%%%%
%%%%% Title of proposal %%%%%
%%%%%%%%%%%%%%%%%%%%%%%%%%%%%

\begin{center}
{\LARGE{\bf
%%
%% ENTER TITLE OF PROPOSAL BELOW THIS LINE
{Are Metallicity Gradients of Spirals Affected by their Dark Matter Halos?}
%%
%%
}}
\end{center}
\bigskip

%% Principal Investigator (PI) initial(s) and family name %%
\centerline{\bf PI:\@
%% ENTER NAME OF PI BELOW THIS LINE
{Elijah Bernstein-Cooper}$^1$}

\smallskip
$^1$University of Wisconsin Madison, USA
\smallskip

% Type a concise abstract of your proposal here (optional).

\section{Abstract}\label{sec:abstract}

    Metals are a fundamental to cooling mechanisms in the intergalactic and
    interstellar medium, star formation, stellar physics, and planet
    formation.  MaNGA will open doors to understanding the chemical evolution
    of spiral galaxies. Metallicity gradients of spirals have been studied
    heavily in small samples and on an individual basis, however, with MaNGA's
    sample of 10,000 galaxies we are presented a unique opportunity to
    understand metallicity gradients across a wide range of galaxy
    characteristics.

    The BOSS spectrographs, each with a red and blue arm will enable
    simultaneous wavelength coverage from 3600\,\AA\ to 10,000\,\AA\ to probe
    sensitive lines to the abundances of heavier elements. The methods
    developed with by the CALIFA collaboration \citep{sanchez12} will guide our
    derivation of the metallicity derivations for a large sample of galaxies.
    The methods of \citet{berg13} will also serve as a useful tool: using the
    temperature sensitive auroral lines [O\,\textsc{iii}] $\lambda$ 4363 and/or
    [N\,\textsc{ii}] $\lambda$5755 to derive metallicity gradients from \hii\
    regions.

    We aim to address how dark matter halos affect the metallicity gradients in
    spirals. We can obtain dark matter halo profiles by dynamical modeling
    \citep[e.g.,][]{cappellari06} and IMF-sensitive spectrophotometric modeling
    \citep{conroy12} as described in the MaNGA overview. We will be able to get
    a total dynamical mass from cross-listed sources with ALFALFA
    \citep{haynes11} to see what percent of the dark matter halo resides near
    the stellar population of the disk.
    
    In particular 
    \citet{martel13} found that the global SFR (especially along the bar) and
    the large-scale flow of enriched gas play a major role in the metallicity
    gradients of spirals.

    \citet{portinari10} found by investigating rotation curves in the framework
    of variable mass to light ratio of stellar discs, they confirm the scenario
    obtained with the constant Mstar/L assumption: a dark matter halo with a
    shallow core, an inner baryon-dominated region, and a larger proportion of
    dark matter in smaller objects.

    \citet{dicintio14} used simulations. The main result is a clear dependence
    of the inner slope of the dark matter density profile on the
    stellar-to-halo mass ratio.

    \citet{dicintio14b} introduce a mass dependent density profile to describe
    the distribution of dark matter within galaxies, which takes into account
    the stellar-to-halo mass dependence of the response of dark matter to
    baryonic processes

    \citet{tissera13} investigate the chemical and kinematic properties of the
    diffuse stellar haloes of six simulated Milky-Way-like galaxies from the
    Aquarius Project. The observed abundance gradients in the inner-halo
    regions are influenced by both the level of chemical enrichment and the
    relative contributions from each stellar subpopulation. Steeper abundance
    gradients in the inner-halo regions are related to contributions from the
    disc-heated and endo-debris stars, which tend to be found at lower binding
    energies than debris stars. 

    \citet{martel13} The main result of this work is therefore that the
    observed enrichment in the centres of barred galaxies is not dominated by
    in situ enrichment by stars formed in the centre. The central metallicity
    does not originate exclusively from central stars. Instead, the global SFR
    (especially along the bar) and the large-scale flow of enriched gas play a
    major role.

    \citet{spavone10} We used high resolution spectra in the optical and
    near-infrared wavelength range to study the abundance ratios and
    metallicities of the HII regions associated with the polar disk in
    NGC4650A, in order to put constraints on the formation of the polar disk
    through cold gas accretion along a filament; this might be the most
    realistic way by which galaxies get their gas. We have compared the
    measured metallicities for the polar structure in NGC4650A with those of
    different morphological types and we have found that they are similar to
    those of late-type galaxies: such results is consistent with a polar disk
    formed by accretion from cosmic web filaments of external cold gas.

    \citet{mott13} We compute the abundance gradients along the disc of the
    Milky Way by means of the two-infall model: in particular, the gradients of
    oxygen and iron and their temporal evolution. First, we explore the effects
    of several physical processes which influence the formation and evolution
    of abundance gradients. They are (i) the inside-out formation of the disc,
    (ii) a threshold in the gas density for star formation, (iii) a variable
    star formation efficiency along the disc, (iv) radial flows and their speed
    and (v) different total surface mass density (gas plus stars) distributions
    for the halo. 

    \citet{fu13} used simulations to show radial gas inflow has little
    influence on gas-phase and stellar metallicity gradients, which are
    affected much more strongly by the fraction of metals that are directly
    injected into the halo gas, rather than mixed with the cold gas. Metals
    ejected out of the galaxy in early epochs result in late infall of
    pre-enriched gas and flatter present-day gas-phase metallicity gradients


    \begin{comment}

%%%%%%%%%%%%%%%%%%%%%%%%%%%%%%%%%%%%%%%%%
%%%%% Body of science justification %%%%%
%%%%%%%%%%%%%%%%%%%%%%%%%%%%%%%%%%%%%%%%%

%% ENTER TEXT, FIGURES AND TABLES BELOW

\section{Scientific Justification}




%-----------------------------Figure Start---------------------------
\begin{figure}[tbh]
    
    \includegraphics[scale=0.3]{figures/sanchez13_fig9.pdf}

    \caption{\em{Figure 9 from \citet{sanchez13}.}\/}

\end{figure}
%-----------------------------Figure End------------------------------


%-----------------------------Table Start-----------------------------
\begin{table}[!ht]
\begin{center}
\caption[]{\em{Here we show the continuum sensitivity required per band.}\/}
\begin{tabular}{cc}
\hline \noalign{\smallskip}
Frequency (GHz) & Sensitivity (mJy) \\
\hline \noalign{\smallskip}
100 & 0.01 \\
300 & 0.10 \\
%\hline \noalign {\smallskip}
\end{tabular}
\end{center}
\end{table}
%-----------------------------Table End ------------------------------

You can structure the scientific justification using the two subsections below (optional).

\subsection{Scientific rationale:}

% Please describe the scientific background of the project,
% pertinent references and previous work relevant to this 
% proposal.

\subsection{Immediate objective:}

% Please describe the observations to be made and their specific
% purpose, with a clear explanation of the need for, and 
% appropriateness of, ALMA Cycle 1 data.  

%%%%%%%%%%%%%%%%%%%%%%%%%%%%%
%% Potential for Publicity %%
%%%%%%%%%%%%%%%%%%%%%%%%%%%%%

\section{Potential for Publicity}

% Here, include a brief statement on the potential of your proposal
% to generate publicity based on the scientific results to be obtained.


%%%%%%%%%%%%%%%%%%%%%%%%
%% References section: %
%%%%%%%%%%%%%%%%%%%%%%%%

\section{References}

% List references here

Sanchez cited here \citet{sanchez12}

    \end{comment}

    \bibliography{my_bib}

%%%%%%%%%%%%%%%%%%%%%%%%%%%
%%%%% End of document %%%%%
%%%%%%%%%%%%%%%%%%%%%%%%%%%

\end{document}

